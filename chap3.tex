\epigraph{연구는 중요하지 않다.\\
          건강이 중요하다.}
 {\textit{2012년 신년교례회에서,\\ \textsc{김웅태 교수님}}}
  
\section{공부에 유용한 서적 소개}
대학원에 입학한 뒤 연구와 관련해 교수님께서 시키시는 일들을 하나하나 해내는 건
그리 어렵지 않다 (정말?).  특히 기술적 문제를 자주 맞딱뜨리게 될 것인데 구글링,
웹상에 떠도는 각종 매뉴얼, 경험 있는 선배들의 도움을 받아 그때그때 해결할 수 있을
것이고, 시간이 지날수록 이러한 어려움은 줄어들 것이다.  그러나 관측/이론을
막론하고 맨땅에 헤딩하는 정신만으로는 해결되지 않는 문제가 있는데 그것은 얕고
좁은 천문학 지식에서 비롯되는 이해 부족이다.  작게는 비슷한 논문을 읽을 때마다
항상 등장하는 용어인데도 그 뜻을 정확히 알지 못하는 경우, 크게는 연구를 시작한지
한참이 지났는데도 자신이 어디쯤 서 있는지 위치 파악이 되지 않는 경우, 모두 이해
부족이라 할 수 있다.  이해 부족은 곧 공부 부족을 뜻한다. 공부 부족인 사람은
(새출발하는 사람이라면 누구나!) 연구하는 시간 외에 따로 짬을 내어 교과서, 논문
등을 스스로 찾아읽어야 한다.  학회나 콜로퀴움에 참석해보면 깨닫겠지만 천문학
연구를 수행할 수 있기 위해서는 다양한 분야의 지식을 필요로 한다.  대부분의 천문학
연구는 관측 데이터를 분석하고 그를 물리적으로 설명하는 것을 목표로 두기 때문에
학부 수준의 역학/전자기학/양자물리/열물리, 그리고 통계학, 광학 등에 대한 폭넓은
이해가 필요하다.

아래에 공부하는 데에 도움이 될만한 책들을 추천한다. Amazon.com의 링크를
포함하였으니 그곳의 리뷰들도 참고하기 바란다.
\begin{packed_item}
\item 일반천문학:
  \href{http://www.amazon.com/Physical-Universe-Introduction-Astronomy-Books/dp/0935702059}{The
    Physical Universe: An Introduction to Astronomy, \textsf{Shu (1982)}}
\begin{packed_item}
\item Frank Shu가 자신의 천문학 강의록을 바탕으로 쓴 책. 천문학의 파인만 강의라고
  불리울 만큼, 출판된지 30년이 지났어도 그 진가를 발하는 속이 꽉찬 책이다.
\end{packed_item}

\item 일반천문학:
  \href{http://www.amazon.com/Introduction-Modern-Astrophysics-Bradley-Carroll/dp/0805304029}{An
    Introduction to Modern Astrophysics (2nd Edition), \textsf{Carroll \& Ostile
      (2006)}}
\begin{packed_item}
\item 천문학과 학부 3-4학년 전공필수 과목의 교과서로 널리 쓰인다. 매우 다양한
  주제를 다루고 (따라서 매우 두껍다!!) 단원마다 연습문제가 있어 기초를 쌓기에
  좋다.
\end{packed_item}

\item 일반천문학:
  \href{http://www.amazon.com/Introductory-Astronomy-Astrophysics-Saunders-Sunburst/dp/0030062284}{Introductory
    Astronomy and Astrophysics (천문학 및 천체물리학 서론), \textsf{Gregory \&
      Zeilik (1997)}}
\begin{packed_item}
\item 일명 천천서로 불리는 학부 서적이다. 이 책은 오히려 번역판을 추천한다.
\end{packed_item}

\item 관측천문학:
  \href{http://www.amazon.com/Observational-Astronomy-D-Scott-Birney/dp/0521853702}{Observational
    Astronomy, Scott \textsf{Birney (2006)}}
\begin{packed_item}
\item 학부 관측 수업 교재로 많이 쓰이는 관측천문학 (이시우 외 옮김)의 2판. 관측
  천문학의 기초를 쌓기에 좋은 책.
\end{packed_item}

\item 외부 은하:
  \href{http://www.amazon.com/Galaxies-Universe-Introduction-Linda-Sparke/dp/0521671868}{Galaxies
    in the Universe: An Introduction, \textsf{Sparke \& Gallagher III (2007)}}
\begin{packed_item}
\item 학부 은하와 우주 수업 교재로 사용됨. 은하 천문학, 외부 은하 천문학의 기초를
  쌓기에 좋은 책.
\end{packed_item}

\item 외부은하 및 우주론:
  \href{http://www.amazon.com/Extragalactic-Astronomy-Cosmology-Peter-Schneider/dp/3540331743}{Extragalactic
    Astronomy and Cosmology: An Introduction, \textsf{Peter Schneider (2006)}}
\begin{packed_item}
\item 우종학 교수님의 대학원 외부은하와 우주론 수업의 교재. 우리 은하, 외부
  은하, AGN, 우주론 등의 관측적 지식을 요약하고, 그들의 정성적 이해에 초점을 두고
  아주 쉽게 설명해준다. 연습문제는 따로 없다.
\end{packed_item}

\item 항성역학, 은하역학:
  \href{http://www.amazon.com/Galactic-Dynamics-Second-Princeton-Astrophysics/dp/0691130272}{Galactic
    Dynamics, \textsf{Binney \& Tremanine (2008)}}
\begin{packed_item}
\item 혹시 모르는 책이라면 링크를 클릭하여 Editorial Reviews만이라도
  읽어보길. 은하역학의 바이블로서 대학원 항성역학 수업의 주교재로도 쓰인다.
\end{packed_item}

\item 천체물리:
  \href{http://www.amazon.com/Radiative-Processes-Astrophysics-George-Rybicki/dp/0471827592}{Radiative
    Processes in Astrophysics, \textsf{Rybicki \& Lightman (1979)}}
\begin{packed_item}
\item 복사론의 표준 교재. 천체물리학 수업의 교재이기도 하다.
\end{packed_item}

\item 천체물리:
  \href{http://www.amazon.com/Astrophysics-Processes-Physics-Astronomical-Phenomena/dp/0521846560}{Astrophysics
    Processes: The Physics of Astronomical Phenomena, \textsf{Hale Bradt
      (2008)}}
\begin{packed_item}
\item 구본철 교수님의 대학원 천체물리학 수업의 부교재로 사용된 책. 천문학의 제반
  물리 현상을 그림을 곁들여 알기 쉽게 설명해준다.
\end{packed_item}

\item 천체물리:
  \href{http://www.amazon.com/Physics-Astrophysics-I-Radiation/dp/0935702644}{The
    Physics of Astrophysics Volume I: Radiation, \textsf{Frank H. Shu (1991)}}
\begin{packed_item}
\item Shu의 천체물리학 책 1권으로 주로 복사론을 다루는, 완성도가 높은 책. 이론을
  전공하는 학생들에게 추천.
\end{packed_item}

\item 천체물리:
  \href{http://www.amazon.com/Physics-Astrophysics-II-Gas-Dynamics/dp/0935702652}{The
    Physics of Astrophysics Volume II: Gas Dynamics, \textsf{Frank H. Shu
      (1991)}}
\begin{packed_item}
\item Shu의 천체물리학 책 2권으로 주로 성간 기체 역학을 다룬다. 역시 이론을
  전공하는 학생들에게 추천.
\end{packed_item}

\item 천체물리:
  \href{http://www.amazon.com/Astrophysics-Gaseous-Nebulae-Active-Galactic/dp/1891389343}{Astrophysics
    Of Gaseous Nebulae And Active Galactic Nuclei, \textsf{Osterbrock \& Ferland
      (1979)}}
\begin{packed_item}
\item Chandrasekhar의 제자인 Donald E. Osterbrock이 원저자이다. HII 영역, 행성상
  성운, 초신성 잔해, AGN 등의 천체가 어떻게 빛을 방출하는지를 상세하게 기술하는데
  복사론, 양자물리에 대한 기본지식은 갖춰야 본문을 제대로 이해할 수 있다. 연구에
  직접 쓰일 수 있는 유용한 결과들을 (예: Line ratio) 표로 제시해준다.
\end{packed_item}

\item 우주론:
  \href{http://www.amazon.com/Modern-Cosmology-Scott-Dodelson/dp/0122191412}{Modern
    Cosmology, \textsf{Scott Dodelson (2003)}}
\begin{packed_item}
\item Fermilab의 저명한 우주론 학자 Dodelson이 쓴 우주론 교재. 표준 모형, 볼츠만
  방정식, 일반 상대론, 인플레이션, 중력 렌즈, 통계적 데이터 분석 방법 등의 다양한
  주제를 다룬다.
\end{packed_item}

\item 우주론:
  \href{http://www.amazon.com/Cosmology-Steven-Weinberg/dp/0198526822}{Cosmology,
    \textsf{Steven Weinberg (2008)}}
\begin{packed_item}
\item 저자에 대해 설명할 필요는 없을거라 생각한다. 저자를 알고 있다면 내용도 굳이
  설명할 필요가 없다?! 이론 우주론을 전공하고 싶다면 필독을 권한다.
\end{packed_item}

\item 통계자료 분석:
  \href{http://www.amazon.com/Reduction-Error-Analysis-Physical-Sciences/dp/0079112439}{Data
    Reduction and Error Analysis for The Physical Sciences, \textsf{Bevington \&
      Robinson (2002)}}
\begin{packed_item}
\item 정확성Accuracy과 정밀성Precision의 차이를 설명할 수 있는가? 데이터의 피팅과
  분석을 어떻게 할 것인지 막막한가? 기술 통계학(Descriptive Statistics)의
  입장에서 데이터 분석의 기본을 가르쳐주는 고전이다. 쉽게 설명 되어있기 때문에
  기초적 개념을 쌓기에 좋다.
\end{packed_item}

\item 기타:
  \href{http://www.amazon.com/Allens-Astrophysical-Quantities-Arthur-Cox/dp/038795189X}{Allen's
    Astrophysical Quantities, \textsf{Arthur N. Cox [Editor] (2001)}}
\begin{packed_item}
\item 천문학에서 많이 쓰이는 물리 상수, 천문학 상수, 단위 변환, 각종
  원자/분자선의 에너지 준위와 파장 등 방대한 양의 정보를 일목요연하게 정리한 책.
\end{packed_item}

\end{packed_item}

박사(博士)의 박(博)은 넓다, 깊다, 많다라는 뜻을 갖고 있는만큼 졸업하기 전까지
독립적 연구자로 성장하여 천문학 연구의 최전선에서 활동할 수 있기 위해서는 폭넓고
깊이 있는 기초 지식을 쌓는 것이 필수라는 생각을 머릿속에 심어두자.  \starbreak

노파심에서 한가지 더 보태고 싶은 것, 대학원생은 공부만 할 수도 없고 연구만 할
수도 없는 신분이다. 배움에 대한 열정이 강한 학생은 수업에 치중하고 위에 열거한
책을 읽느라 연구를 소홀히 하기 쉽고, 하루빨리 연구 성과를 내는 것을 유일한 목표로
둔 학생은 공부를 소홀히 하기 쉽다. 각 시기마다 연구와 공부의 균형점을 잘 찾아
공부/연구하는 시간 배분에 신경쓰도록 하자 (시간이 지날수록 연구에 집중해야함은
물론이다).

\section{문헌 검색}
어떤 연구를 시작할 때는 해당 주제 또는 관련 주제에 관해 이미 출판된 일들을
찾아보는 일이 우선되어야 한다. 이를 위해서 문헌 검색이 필요하다. 여기서는 천문학
논문을 찾을 때 주로 쓰게 되는 몇 가지 도구들을 소개한다. 더불어 천문학은 데이터
의존성이 높은 학문으로 한번 측정된 값은 두고두고 가치가 있다. 그렇기 때문에
천문학 커뮤니티에서는 측정한 값들을 카탈로그화/데이터베이스화 하는 일이
활발하다. 여기서는 천체에 관한 정보를 찾아볼 수 있는 대표적인 데이터베이스 NED와
SIMBAD를 소개한다.

\subsection{ADS}
천문학 문헌 검색의 가장 대표적이고 자주 사용하게 되는 데이터베이스는
Astrophysics Data System (ADS, \url{http://adsabs.harvard.edu/})
이다. Smithsonian Astrophysical Observatory (SAO)에 의해 운영되는 이
데이터베이스는 9천 3백만 편이 넘는 천문학, 물리학 논문을 검색할 수 있다. 뿐만
아니라 NED, SIMBAD 데이터베이스와 연계되어 있어 해당 논문에 나오는 천체들에 관한
데이터도 바로 열람할 수 있다.

ADS에서는 다양한 검색 기능을 제공한다. keyword를 가지고 간단하게 검색해볼 수도
있지만 저자, 출판날짜, 제목, 초록에 들어가는 말을 구체적으로 지정하여 검색할 수도
있다. 이에 관한 full documentation은 해당 사이트 help page에서 볼 수
있다. 여기서는 간단한 검색 문법 몇 가지를 소개한다.

\begin{tabular}{c|c} \toprule
문법 & 목적 \\ \midrule
\texttt{$\char`\^$Smith, J.} & as the first author \\
\texttt{Smith, J.} & as the last author \\
\texttt{Smith, J.\#} & author without any middle name \\
\texttt{M1?} & one wildcard character (matches M10 or M19, etc.) \\
\texttt{3C*} & zero or more wildcard characters (matches all 3C objects) \\
\texttt{``a b'', a.b, a--b} & 구(phrase)를 순서 그대로 검색 \\ \bottomrule
\end{tabular}

\vspace{\baselineskip} 단어 조합의 경우 단순히 AND/OR을 선택할 수 도 있고, 어떤
단어를 포함하고 어떤 단어 포함하지 않는 논문을 검색할 것인지 지정할 수
있다. simple logic 방법을 택하면 각 단어 앞에 +/-를 붙여 해당 단어를
포함하는/하지 않는 논문을 검색할 수 있다. 더 복잡한 조건을 원한다면 boolean
logic을 이용해 괄호/AND/OR을 포함한 표현으로 검색을 할 수도 있다. ADS에서는 검색
결과를 점수(score)를 가지로 정렬해주기도 하는데 이 점수는 어떤 검색 기능을
사용하느냐에 따라 의미가 조금 달라질 수 있다.
 
\subsection{arXiv}
ADS와 더불어 가장 유용하게 사용할 수 있는 문헌 검색 도구로는
arXiv(\url{http://arxiv.org})가 있다. arXiv는 1991년에 시작된 자동화된 논문 제공
시스템으로 다양한 분야에서 매일 새로운 논문들이 올라온다. 이를 잘 활용하면 저널에
출판되기 전에 가장 빨리 최신 연구 논문들을 무료로 읽을 수 있다. 논문 뿐 아니라
학회 프로시딩이나 강의 노트 등이 올라오기도 한다. \textbf{astro-ph new
  submission의 주소 (\url{http://arxiv.org/list/astro-ph/new})를 즐겨찾기 해두고
  매일 매일 방문해서 최소한 흥미있는 논문의 제목, 초록 등을 읽는 것을 습관화하는
  것}을 강력히 추천한다.

\starbreak ADS나 arXiv와 더불어 Google scholar나 Wikipedia등의 서비스도 잘
활용하면 내가 원하는 정보를 쉽게 얻을 수 있다. 그러나 그 정보의 신뢰성은 언제나
스스로 판단하여야 함을 유의하자. 인터넷을 통해 찾을 수 있는 문헌 이외에도 교내
중앙도서관(\url{http://lib.snu.ac.kr})에서는 다양한 학술 저널들을 열람하거나
복사할 수 있는 서비스를 제공한다. 더불어 규모는 작지만 학과 건물에 있는 천문
도서관 (19동 301호)에서도 연구에 유용한 책들을 대출할 수 있다.

\subsection{Astronomical Databases} 
천체에 관한 측광, 분광, 위치, 문헌 정보를 검색해보고자 한다면 NASA Extragalactic
Database (NED, \url{http://ned.ipac.caltech.edu}) 나 SIMBAD Astronomical
Database (\url{http://simbad.u-strasbg.fr/simbad}) 를 이용할 수 있다.

\section{연구자료 및 문헌 관리}
\subsection{논문 정리}
천문학 연구를 하면서 다양한 배경지식을 습득하고 다른 연구자들의 방법을 알기
위해서는 많은 논문을 읽어야 한다. 방대한 논문들을 읽어가다 보면 읽은 논문들을
일일이 기억하지 못하기 때문에 다시 찾아봐야 할 때가 많다. 뿐만 아니라 참고문헌에
들어가야 할 중요한 논문들은 따로 보관한다면 좀 더 편리하게 논문을 작성할 수 있을
것이다. 그렇다면 매일 늘어만 가는 논문들을 어떻게 정리해야 할 것인가. 논문 정리에
있어서 정답은 없을 것이다. 무엇보다 자신에게 적합한 관리방법을 찾아 효율적인
연구를 하면 좋을 것이다. 다양한 논문 관리 방법들이 다음에 제시된다.

\begin{description}
\item[읽고 요약하기] \hfill \\
  논문을 잘 정리 해 둔다고 하여도 다음에 볼 때 자세한 내용은 기억이 잘 나지 않아
  몇 번씩 다시 읽어야 할 때가 많다. 논문을 읽을 때마다 데이터, 연구방법, 결과
  등을 잘 요약해두면 논문 내용을 상기시키는 데 많은 도움이 될 수 있다. 아래의
  논문 정리방법과 함께 병행하면 효과가 배가 될 것이다.
\item[ADS 이용] \hfill \\
  3.1장의 논문 검색에서 ADS에 대하여 알아보았다. ADS는 논문 검색만이 아니라 찾은
  논문들을 정리할 수 있는 기능이 있으므로 이 기능을 유용하게 이용한다면 논문을
  찾을 때 바로 정리를 같이 할 수 있으므로 따로 정리하는 번거로움이 줄어들
  것이다. 우선 ADS 홈페이지에서 오른쪽 상단 ‘sign on’ 을 누르면 email 주소로
  간단히 계정을 만들 수 있다. 계정을 만든 뒤 계정에 접속하여 논문을 검색하다가
  저장해두고 싶은 논문 하단에 ‘Add this article to private library’ 을 누르면
  손쉽게 논문을 저장해둘 수 있다. 또한 Library를 여러 개 만들어 논문을 주제 별로
  정리할 수 있다. 논문을 볼 때 마다 ADS에 접속해 논문을 다운받아야 하는 불편함이
  있지만 개인 PC를 포맷하여도 논문들을 잃어버릴 염려가 없다는 장점이 있다.
\item[개인 PC에 저장] \hfill \\
  개인 PC에 논문을 저장하는 방법은 무난하여 가장 많이 쓰는 방법이다. 하지만
  폴더를 자세히 분류하지 않고 저장만 해놓는다면 PC에 논문이 저장되어있더라도
  논문을 다시 검색해야 하는 불편함이 생길 수 있다. 개인 PC에 정리해 둘 때도
  자신만의 방법으로 저장하는 것이 좋다. 우선 한 폴더 안에 논문의 개수가 적도록
  폴더를 자세하게 만들거나 파일의 이름을 논문 제목으로 저장한다면 논문을 손쉽게
  찾을 수 있을 것이다. 또는 논문의 파일명, 제목을 정리한 파일을 만드는 것도
  편리하다. 이때 (1)에서 언급했었던 논문 요약도 이 파일에 같이 정리해도 좋을
  것이다.
\item[다양한 문헌 관리 툴] \hfill \\
  위에서 설명한 방법 외에도 다양한 문헌관리 툴이 있다. 많은 종류의 문헌 관리 툴을
  모두 설명하기는 힘들기 때문에 여러 가지 툴을 언급하고 넘어갈 것이다. EndNote는
  사람들이 가장 많이 쓰는 상용 논문 관리 소프트웨어다. 논문의 제목, 저자,
  초록(abstract), 저널명, 연도, 페이지 등을 기록해 둘 수 있으며, pdf파일의
  URL링크나 내 하드디스크의 링크를 걸 수 있다. 구축된 목록은 워드/엑셀등과
  연동되어 논문을 작성하거나, Reference를 만들 때 편리하게 쓰인다. bibDesk는 같은
  기능을 하는 무료 소프트웨어다. bibTex의 GUI버전이다.

  EndNote, Papers, BibDesk의 다운로드와 자세한 설명은 아래의 링크를 참고.
\begin{packed_item}
\item EndNote
\begin{packed_item}
\item \url{http://www.endnote.com} (공식홈페이지 및 다운로드)
\item \url{http://medlib.korea.ac.kr/html/endnote/EndNote_Guide(Basic).ppt}
  (자세한 사용법)
\end{packed_item}
\item Papers
\begin{packed_item}
\item \url{http://www.mekentosj.com/papers} (공식홈페이지)
\item \url{http://zootcool.tistory.com/250} (자세한 사용법)
\end{packed_item}
\item BibDesk: \url{http://bibdesk.sourceforge.net} (공식홈페이지 및 다운로드)
\item Mendeley : \url{http://www.mendeley.com} (공식홈페이지 및 다운로드)
\end{packed_item}

\end{description}

\subsection{연구자료 관리}
연구를 하다 보면 수많은 데이터를 처리하고 그림을 그리고 프로시저를 만들게
된다. 서버와 개인용 PC에 이상이 생기거나 천재지변에 대비하여 서버나 외장하드를
이용하여 백업을 자주 해두는 습관이 중요하다. 작업한 데이터의 용량이 너무 커
백업을 하지 못할 시에는 적어도 처리하기 전의 관측 데이터와 스크립트, 개인이 만든
프로시저만이라도 백업을 해둔다면 만일의 사태에도 연구를 신속히 다시 수행할 수
있을 것이다.

물질적인 데이터의 백업도 중요하지만 연구가 어떻게 수행되었는지를 기록하는 것도
중요하다. 데이터를 잃어버린 것과 마찬 가지로 연구 방법 또는 연구결과물의 위치
등이 기억나지 않는다면 연구를 다시 수행하여야 하기 때문이다. 연구 기록 방법에
대해 2가지만 제시하도록 하겠다.

\begin{description}
\item[연구노트 작성] \hfill \\
  매일 연구노트를 작성하여 기록하는 방법이다. 연구노트는 일기처럼 매일 한 연구를
  정리하여 적어두는 노트이다. 오늘 한 일에 대하여 당장은 기억이 나지만 비슷한
  작업을 반복하다 보면 몇 달 뒤 본인도 깜짝 놀랄 정도로 기억나지 않는 것들이
  분명히 있을 것이다. 그렇기 때문에 연구노트는 자세히 쓸수록 좋으며 이것이
  힘들다면 중요한 부분만이라도 요약하여 (어떤 일을 수행하였고 결과물의 이름과
  저장경로) 정리하는 것도 도움이 된다.
\item[작업공간에 정리] \hfill \\
  연구노트는 시간 순으로 정리가 되기 때문에 여러 가지 일을 병행하는 사람이라면
  주제 별로 뒤죽박죽되어 정리가 잘 되지 않을 수도 있다. 작업공간에 바로 정리를
  해둔다면 연구 주제별로 정리를 할 수 있다. 가령 폴더 마다 ‘README'와 같은 파일을
  만들어 그 폴더 안에 들어있는 데이터, 프로시저, 다른 모든 파일에 대한 설명을
  써주면 용이할 것이다. 자주 업데이트를 해주지 않으면 파일에 대한 예전 설명으로
  혼동을 줄 수 있으니 업데이트를 자주 하도록 노력하자. 파일에 따른 설명을 일일이
  적어두기 귀찮다면 자주하는 작업들에 대해서는 결과 파일의 이름을 동일한 기준을
  적용시켜 그 기준만 적어두어도 좋다. 폴더와 파일명은 기억하기 쉽도록 하는 것이
  중요하지만 너무 간단하면 중복되기 쉽고 복잡하면 이름이 길어지므로 적당한
  자신만의 작명법에 대해서 고민해보는 시간은 연구 과정을 기억해내는 데 쏟는
  시간에 비하면 아깝지 않은 시간이라고 생각한다.
\end{description}

이상 논문정리와 연구 내용 관리에 대하여 살펴보았다. 궁극적인 목표는 연구를
효율적으로 잘 수행하기 위한 것임을 명심하면서 수단에 집착할 필요 없이 자신만의
효과적인 방법을 찾는 것이 가장 좋을 것이다. 다만 처음부터 조금씩 정리하지 않으면
나중에는 정리할 수 없을 만큼 방대해지니 필요성은 인식해두면 좋겠다.

\section{연구에 사용되는 각종 도구}
이 장에서는 여러분이 앞으로 천문학 연구를 하게 됨으로써 기본적으로 사용하게 될
도구들에 대해 간략히 소개하고 유용한 레퍼런스들을 제공하여 스스로 공부하고
익히는데 도움이 되고자 한다. \url{astro.snu.ac.kr/~grad}에 유용한 레퍼런스의
링크를 모아두었으니 적극 활용하도록 하자.

\subsection{Operating System}
천문학 연구를 위한 데이터 처리 및 분석에 적합한 환경으로 Linux 또는 Mac OS X
운영체제를 추천한다. 여러분은 앞으로 터미널 환경에서 많은 작업을 수행하게 될
것이다. 따라서 Windows 운영체제는 우리의 연구 작업 자체를 수행하는데 있어 다소
비효율적이고 부적합한 측면이 있다. (작업 자체를 windows에서 하는 것은 비추; 물론
윈도우 자체로 유용한 점 있음 ppt라든지 한글문서라든지... 아 물론 작업은 모두 팀
서버(보통 리눅스 클러스터)에 접속해서 한다면 클라이언트로 뭐를 쓰건 크게 상관은
없겠다.)

\begin{packed_item}
\item 커멘드 라인 모드의 강력함/편리함에 익숙해져야
\item 리눅스 배포판 하나 설치 해보기 (fedora, ubuntu ...)
\item 맥에는 이미 BSD(버클리 대학에서 개발한 유닉스 계열 운영체제)가 깔려 있음
\end{packed_item}

\subsection{Text Editor}
여러분은 앞으로 수많은 스크립트/코드들과 마주하게 될 것이다. 이미 존재하는
코드들을 보든 스스로 짜든 상당히 많은 시간을 코드 작성과 수정에 할애하게 될
것이다. 따라서 연구 수행의 많은 시간을 텍스트 에디터와 함께 하게 되므로 편리하고
효율적인 좋은 툴을 처음부터 선택하여 활용하는 것이 향후 연구 작업 능률 향상에
매우 큰 영향을 끼칠 것이다. 에디터의 선택은 어느 정도 개개인 취향의 문제이나 vim
이 기본 중의 기본이라 할 수 있겠다. (물론 윈도우즈 사용자들에게 익숙한
인터페이스를 가지는 여러 유용한 GUI 기반의 좋은 에디터들이 존재 한다: 예)
통합개발환경을 지원하는 멋진 에디터 eclipse)

기본적으로 vim(Vi IMproved)을 추천한다.
\begin{packed_item}
\item 유닉스 세상의 전통을 이어온 가장 널리 사용되는 에디터
\item 키보드로 모든 조작이 가능한 가볍고 빠른 효율적인 강력한 에디터
\item command mode/insert mode/ex mode의 세 가지 모드로 구성
\end{packed_item}

또는 emacs(Editing MACRoS)를 추천한다.
\begin{packed_item}
\item 단순한 편집기를 넘어서는 텍스트 처리를 위한 포괄적인 통합환경, 또는 응용
  프로그램 실행 환경을 제공
\item 프로그래밍에 보다 최적화(각 언어용 모드 존재)
\item 엄청나게 많은 키 명령과 vim보다 훨씬 강력한 확장성을 제공함
\end{packed_item}

\subsection{Data Reduction Package}
천문학 연구에서 관측 기기를 통해 관측한 raw data를 과학적 분석을 위한 수준의
데이터로 만드는 데는 여러 가지 전처리 과정이 필요하다. 이 과정을 data reduction
이라고 하며 우리는 주로 IRAF(Image Reduction and Analysis Facility)를
사용한다. IRAF를 사용한 측광/분광 데이터 리덕션에 대해 이미 선배들이 만들어둔
매뉴얼들을 참고하라.
\begin{packed_item}
\item 황호성 박사: \url{https://www.cfa.harvard.edu/~hhwang/Manuals.html}
\item 이준협 박사: \url{http://astro.snu.ac.kr/~jhlee/}
\item 박근홍: \url{http://astro.snu.ac.kr/~khpark/}
\end{packed_item}

\subsection{Programming Language}
앞으로 여러분은 많은 시간을 컴퓨터 프로그래밍 언어에 익숙해지고 그것을 사용하는데
할애하게 될 것이다. 수치계산, 데이터 분석, 시뮬레이션, 분석결과로 그래프/이미지
생성 등에 프로그래밍 언어를 사용하게 되는데, (천문학 연구에서 주로 사용되는)
다음과 같이 크게 2가지 타입의 언어가 있다. (해당 소속 연구실에 따라 주로 사용하는
언어가 달라질 텐데 기본적으로 IDL 만큼은 사용법을 알아두는 것을 추천한다.)

\begin{itemize}
\item Fortran, C: 컴파일러 언어. 주로 빠른 계산 속도가 필요하고 병렬화가 필요한
  수치 시뮬레이션에서 많이 쓰임. 방대한 양의 유용한 라이브러리들 존재.
\item IDL: 인터프리터 언어. 과학 연구자(특히 천문학자)를 위한 자료 처리 및
  분석/가시화 언어, 폭넓은 천문학자 유저 층과 풍부한 천문학 관련 라이브러리
  존재, 현재 서울대 천문학과가 사이트 라이센스를 사서 보유; 추천! :)
\item Python(파이썬): 오픈소스로, 보다 진화된 프로그래밍 언어, 훨씬 다양한
  분야에서 널리 활용됨. 최근 천문학계에 유저 층이 늘어나고 있음, 앞으로 다양한
  천문학 프로젝트에 많이 적용되어 쓰일 듯, 이미 존재하는 여러 다양한 패키지에
  대해 glue 또는 wrapper 언어로서 앞으로 중심에 설 듯.
\end{itemize}

\subsection{LaTeX}
천문학 연구의 최종 목적은 결국 논문 출판이다. 자신의 연구 결과를 하나의 논문으로
작성 하는데 우리는 LaTeX을 사용한다. LaTeX은 문서 작성 시스템 (오픈소스) 으로
논문/책/매뉴얼/발표슬라이드 등의 작성에 활용할 수 있다.\footnote{이 매뉴얼 또한
  LaTeX을 이용해 만들어졌다.} 무엇보다도 아래의 레이텍 입문서를 한번은 읽어 볼
것을 강력히 추천한다. LaTeX의 설치와 사용에 대한 좀 더 많은 정보들은
\url{astro.snu.ac.kr/~grad}에 올려둔 문서에서 가르쳐주는 각종 인터넷 정보들을
활용하도록 하자.

\section{학회의 종류 및 학회 참석 노하우}
네이버 백과사전에서 “학회”라고 검색을 하면 아래와 같은 학회의 정의가 나온다.

\begin{quote}
\textbf{학회.} \textit{학문을 깊이 있게 연구하고 더욱 발전하게 하기 위하여 공부하는 사람들이 만든 모임.}
%  \hspace*{\fill} 
\end{quote}
즉, 학회에 참석하는 우리는 ‘천문학’ 이라는 학문을 발전하게 하기위하여 무언가
노력을 해야 한다는 사실을 잊으면 안 된다. 적어도 학회장에서 만큼은.. 절대
과자하고 저녁만찬 먹으러 가는 곳이 아니다.

학회는 전공분야와 참여 구성원 등을 기준으로 여러 갈래로 나눌 수 있지만 크게
국내학회와 국제학회로 나눌 수 있다. 4개국 이상이 참여하며, 구두발표 논문이 20건
이상이고, 외국기관 소속 외국인이 50\%이상일 때 국제학회로 분류 된다. 2011년 한 해
동안 국내에서 개최된 천문학 관련 학회의 수는 수십 개가 넘고, 또 전 세계에서 열린
천문학 관련 국제학회의 수는 300개에 육박한다. 본 장에서는 이처럼 많은 학회 중에서
서울대학교 천문학과 대학원생들이 중요하게 여겨야 할 학회들 몇 가지와, 학회에
참석할 때 알아두면 좋은 Tip 몇 가지를 소개하도록 한다.
 
\subsection{학회의 종류}
\subsubsection{한국 천문학회 (The Korean Astronomical Society)}
한국 천문학회는 천문연구원에서 주관하는 국내 최대의 천문학회로 1965년 3월 21일 에
창립되었다. 학회가 처음 생기던 때에는 20명의 회원밖에 없었으나 근래에는 600명까지
늘었다고 한다. JKAS, PKAS, BKAS 이렇게 3종의 학술지를 발간한다. JKAS는 Journal
of the Korean Astronomical Society의 약자로, 1년 6회, 영문으로만 발간하는
학술지이며, PKAS는 Publications of the Korean Astronomical Society의 약자로, 1년
2회 이상 발행하는 것을 원칙으로 하고 있다. “천문학 논총”이라는 책이 바로
PKAS다. 마지막으로 BKAS는 Bulletin of the Korean Astronomical Society의 약자로
천문학회에 등록한 사람들에게 배부하는 노란 책이다. 학술대회 발표초록, 기사, 학계
동향 등이 수록되어 있다. 이들 학회지는 연구실마다 무수히 굴러다니고 있으므로
심심할 때마다 한 번씩 읽어보도록 하자.

한국천문학회는 1년에 2번, 봄 (4월 중순)과 가을 (10월 중순)에 학술대회를
개최한다. 짧게는 1박 2일, 길게는 2박 3일 동안 진행되며 100개 내외의 구두,
포스터발표가 선보이다. 전국에 있는 박사님들과 대학원생들이 한 장소에 모여야 하는
만큼 장소선택은 민감한 사항인데, 통계적으로 수도권 1회, 비수도권 1회씩 각
대학교나 리조트에서 개최한다. 자신의 연구를 발표하기 위해서는 학술대회가 시작하기
40~50일 전에 초록을 제출해야 하므로, 미리 알고 연구에 박차를 가하도록 하자.

\subsection{학회 참석 노하우}

\subsubsection{학회에 참석하기 전에}
학회에서 본인의 발표가 없다면 상대적으로 여유로운 시간을 보낼 수 있을
것이다. 그렇다면 학회에서 어떠한 연구가 발표되는지 미리 확인하고, 관심 있는
연구가 있을 경우 그 분야에 대해 미리 공부해 놓도록 하자. 관심 있는 연구가 이미
논문으로 발표되었다면 당연히 미리 읽어야 한다. 발표된 논문이 없더라도 저자의 최근
연구나 학계의 비슷한 연구를 미리 공부해 간다면 학회장에서 발표를 이해하는데 매우
큰 도움이 된다. 짬이 어지간히 차지 않는 이상, 대학원생이 학회에서의 발표를 한
번에 이해하기는 어렵기 때문이다. 본인이 학회에서 발표를 한다면 참석하기 며칠
전부터 연습에 연습을 거듭해야 한다. 발표를 잘하기 위한 요령에 대해서는 3.6장
발표요령에서 다루기로 한다.  학회가 우리학교에서 개최되지 않는 이상, 버스, 기차,
운이 좋으면 비행기를 타고 멀리 가야한다. 그런데 이게 생각보다 피곤하다. 개인차가
있겠지만, 매일 의자에만 앉아서 연구하다가 갑자기 어디론가 가야한다는 것은 알게
모르게 몸에 스트레스를 준다. 이렇게 학회 당일 꾸역꾸역해서 학회장에 도착하면,
몸은 피곤한데 넓고 조용한 학회장에서 어려운 연구내용을 (그것도 때론 영어로)
발표한다. 열심히 경청해야 하는데, 사람인지라, 나도 모르게 졸음이
쏟아진다. 그러니까 학회가 시작하기 전날 미리 도착해 준비하거나, 아니면 그 전날
잠을 많이 자 두도록 하자.

\subsubsection{학회장에서}
대부분 학회에서는 학회가 시작하기 전에 발표제목과 초록을 정리해둔 책을
배부한다. 본인의 경우, 이 책을 받으면 다음과 같이 행동한다. 먼저 발표목록을 쭈욱
보면서 평소에 관심 있거나, 혹은 재밌을 것 같은 제목에 빨간펜으로 밑줄을
긋는다. 그럼 내가 어느 세션에 들어가야 하는지가 대략 결정된다. 해당 세션 장으로
자리를 옮긴 후, 곧이어 있을 발표의 초록을 천천히 읽는다. 초록의 중요문장에 밑줄도
긋고, 물어보고 싶은 점도 미리 생각해 놓는다. 해보시라. 천지차이다.  학회의 가장
큰 장점은 평소 궁금한 내용을 직접 질문\&논의할 수 있다는 것이다. 국내외 학자들이
먼 길을 마다하고 한 장소로 모이는 바로가 바로 이것 때문이다. 즉 학회장에서는
평소에 만나기 힘들었던 대가에게 직접 질문하고, 연구방향에 대한 조언을 얻어야
한다는 사실을 잊지 말자. 본인이 포스터 발표를 한다면 미리 A4용지에 몇 장 여분을
프린트 해가도록 하자. 그리고 학회장에서 비슷한 분야를 연구하시는 박사님들께 한
장씩 드리면서 연구에 대해 논의해 보자. 좋아하신다. 많이.

\subsubsection{학회가 끝난 후에}
세션이 끝난 이후에는 10분~20분 정도의 쉬는 시간이 있다. 이 시간에는 과자만
쳐묵쳐묵하지 말고 세션을 들으면서 궁금했던 내용들을 선배들이나, 다른 박사님들과
이야기 하면서 이해를 높이도록 하자. 그리고 관심 있었던 연구의 초록을 다시 보면
발표내용이 깔끔하게 정리될 것이다. 이렇게 하면 학회가 끝나고 연구실에 돌아와도
발표내용이 머릿속에 오래 남아 큰 도움이 된다. 학회가 당일로 끝나는 것이 아닌 1박
이상이 된다면 밤에는 자연스레 술과 함께하는 다과회가 이어진다. 함께 연구할 인맥을
만든다는 좋은 취지는 개뿔.. 이해하지만 너무 많이 마시지는 말자. 다음날 세션에
지장 있다 (맨정신으로도 이해 못하는거 술 안 깨면 어쩔꺼임?)

\starbreak   

물론 본인은 지금도 위에서 열거한 사항들을 다 지키지 못한다. 하지만 알면서도 못
지키는 것과 아무것도 모르고 지키지 못 하는 것은 큰 차이가 있기에, 본인이 3년 동안
학회에 참석하면서 생각했던 내용들을 기술해 보았다. 앞으로 우리 천문학과에 새로
입학할 대학원생들은 본인보다는 훨씬 올바른 대학원 생활을 할 수 있을 거라는 기대를
안고서 짧은 글을 마친다.
 
\section{발표 요령}
누구나 발표는 한번쯤 해봤을 거라 생각한다. 적어도 수업시간에 했던 조별 발표라도
말이다. 다들, 발표는 잘 했었는지?

연구는 자기만족을 위해서 할 수 있다. 하지만 발표는 아니다. 발표를 한다는 것은
자신이 공부한 것, 연구한 것들을 남들에게 보여주는 거다. 내 연구는 내가 제일 잘
안다. 남들이 “네가 하는 연구는 내 관심사가 아니야”라고 해도 할 말 없다. 내가
재미있으면, 내가 궁금하면 그것으로 연구를 하는 이유는 충분하다. 그러나! 내 발표가
너무 재미가 없어서, 혹은 너무나 미천...해서 사람들이 듣는 둥 마는 둥 한다면? 그건
실패한 발표다.

발표의 목적은 그런 거다. 자신의 연구를 ‘광고’하는 역할. “내가 얼마나 중요한 일을
하고 있는데 말이야, 쉽게 얘기해 줄 테니 잘 들어봐!”하는 배짱. 이런 것이
발표다. 이 장에서는 이러한 발표의 종류와, 준비하는 방법, 요령 등을 말해주고자
한다. 나도 잘 못하지만, 그래도 이러한 요령들 덕분에 그나마 많이 나아졌다고 할 수
있으니... 관심을 가져보시길.

\subsection{발표 유형}
보통 학부시절에 했던 발표들을 떠올리면 대부분, 파워포인트를 만들고 사람들 앞에
나가서 레이저를 쏴 대면서 하던 발표가 생각날 것이다. 하지만 대학원생들이 학회나
워크샵에서 하는 발표는 이것만 있는 게 아니다. 발표의 두 종류, 구두 발표와 포스터
발표에 대해서 간략히 이야기 하겠다.

\subsubsection{구두 발표}
보통 수업 시간에 발표했던 방식을 생각하면 된다. 자기가 한 연구를 직접 말로
설명하는 거다. 입맛에 맞게 파워포인트든 키노트든, 프레젠테이션 준비를 해야 하고,
발표 시간 또한 정해져 있다. 보통 학회에서는 15분 정도의 시간을 지정해주는데,
발표와 질문 시간 포함 15분이므로 시간을 잘 배분하는 것이 좋다.

15분은 자신이 한 연구를 소개하기에 참 짧은 시간일 수 있다. 논문을 써서 출판하는
것보다 훨씬 압축적으로 청중에게 연구를 소개하고, 관심을 훅! 끌 수 있어야
한다. 이것이 바로 구두 발표의 어려운 점.

\subsubsection{포스터 발표}
포스터 발표는 생소하게 느껴질 거다. ‘포스터’하면 초중딩 때 그렸던 불조심 포스터
이런 거 생각날 테니. 포스터 발표는, 전지나 전지 반 정도 크기의 종이에 자신의
연구를 요약해서 적은 것을 학회장이나 세미나장에 붙여 놓는 발표 형식을
말한다. 말로 발표하는 것보다 더 자세히 적어놓을 수 있고, 15분으로 제한된 구두
발표에 비해 거의 학회 내내 붙여 놓을 수 있기 때문에 사람들의 이해도를 높일 수
있는 장점이 있다.

그러나 이러한 방식은, 가만히 앉아 있기만 하면 되는 구두 발표의 청중들과는 달리
수고롭게도 사람들이 직접 그 포스터 앞에 와서 읽어야 하는 단점이 있다. 따라서
효과적인 전달을 위해서는 오히려 구두 발표보다 더 노력이 필요할 수 있다. 학회 쉬는
시간에도 포스터 앞에 대기해야 하고, 자신의 분야를 연구하는 사람들을 직접 대면해야
하기 때문이다.

\subsection{발표 준비하기}
이제 발표의 두 종류에 대해서 알았으니, 어떻게 준비해야 하는지에 대해서
적어보겠다. 기본적인 사항들만. 사실 다른 사람들의 발표를 직접 들어보는 것이, 준비
과정을 배우는데는 훨씬 효과적일 수 있다. 그러니 남의 발표도 잘 듣도록.

\subsubsection{구두 발표}
발표를 준비할 때 보통 들어가야 할 내용들이 있다. 논문에도 들어가는 내용들인데,
다음과 같다.

\begin{quote}
\begin{center}
\textsf{Introduction, Method, Result, Discussion, Summary}
\end{center}
\end{quote}
꼭! 위와 같은 제목을 쓸 필요도 없고, 순서를 꼭 따라야 하는 것도 아니다. 자신이 중요하게 생각하는 순서에 따라, 사람들이 받아들이기 편할 것 같은 순서에 따라 개요를 작성하면 된다. 예를 들어, 자신이 중요하다고 생각하는 연구 결과들이 있으면, 그 결과들을 큰 카테고리로 하여 위의 내용들을 잘 버무려도 된다는 말이다. 어떻게 조합해서 발표하는가에 따라 훨씬 매력적으로 발표를 꾸려나갈 수도 있겠다.\\

그렇다면 들어가야 할 내용들에 대해 좀 더 자세히 보도록 하자.

Introduction에서는 이 연구가 왜 중요한지, 목적이 뭔지에 대해서 설명해야
한다. 자신의 연구를 소개하는 입장이므로 이 부분을 확고하게 밝힐 수 있어야
한다. 여기서 흥미를 유발해야지 사람들이 학회장에서 도망가는 불상사를 막을 수
있다.

여기서 잠깐. 흥미를 유발하려면? 쉽게 준비해야 한다. 청중의 수준을 너무 높게 잡지
마라. 내 연구를 모르는 사람도 알아듣도록 준비해야 하는 거다. 그 연구에 대해 잘
아는 사람도 전문 용어가 난무한 발표보다는, 알아듣기 쉽게 이야기하는 발표를 좋아할
것이다.

15분의 짧은 시간을 잘 배분하기 위해서는, Method 부분을 간략히 하는 편이
좋다. 사람들이 관심 있는 부분은 ‘왜 이 연구를 했는지, 그래서 무슨 결과가
나왔는지, 그 결과를 어디다 써먹을 수 있는지’이기 때문에 방법 측면은 최대한
간결하게 핵심만 이야기하고 넘기는 편이 좋다. 사실 대학원생들이 했던 일들은 대부분
Method에 들어갈 내용이기 때문에, 자기가 했던 모든 잡일들을 다 써 넣고
싶겠지만... 그걸 생략하자니 한 게 없어 보이고 그렇겠지만... 과감히 중요한 것만
이야기 하도록. 진정 그 과정이 궁금한 사람이라면 직접 물어보게 되어 있으니.

Result와 Discussion에서는 자신의 연구 결과를 보여주면 된다. 그것을 해석하고, 큰
그림 상에서 어떤 중요성을 가지게 될 것 인지까지 포함하여 이야기하는 것이 좋다. 이
두 부분 또한 Introduction과 마찬가지로 자신의 연구가 얼마나 타당하고 획기적인
결과인지를 주장하는 부분이므로, 자세히 이야기하자.

이러한 내용들이 다 포함되도록 발표를 알차게 준비했다면, Summary 슬라이드를
만들자. Summary는 Introduction부터 Discussion까지의 모든 내용을 슬라이드 하나에
요약하는 것이다. 발표가 끝나고 사람들이 이 부분을 읽으며 다시 되새길 수 있도록
준비하는 것이 좋다. 발표 마지막 슬라이드에 ‘Thank you~’와 같은 문구만 덜렁
적어놓는 것보다 훨씬 유익하다.

자자, 발표 절차에 따른 프레젠테이션 파일이 만들어졌다. 이제 확인해야 할 것이 하나
있다. 다시 앞으로 돌아가서 차근차근 자신이 쓴 내용들을 읽어봐라. 뭐가 빠지진
않았나? Reference가 제대로 적혀 있는지 확인해라. 내가 무슨 데이터를 썼는지, 무슨
연구에서 동기부여를 받았는지를 명확히 쓰는 거다. 그림 하나, 사진 하나를 따 와도
그 출처를 밝혀야 한다. 이는 기본적인 태도에 관련된 사항이므로 주의할 것.

이제 진짜 다 만들어졌다. 그럼 이제 학회장에서 바로 발표만 하면 될까?
아니다. 리허설을 해야지. 시간 맞추는 게 엄청 중요하기 때문에, 직접 발표 연습을
하면서 시간을 맞춰봐야 한다. 시간 내에 끝낼 수 있는지, 해야 할 말은 다 했는지
점검하며 리허설을 한다. 연습 한 것과 안 한 것은 엄청난 차이가 있다. 긴장을 덜 수
있기 때문이다. 연습한 대로만 하면 되는 거다.

그리고! 리허설은 혼자 하면 안 된다. 팀 선배들을 부르든, 다른 동료를 부르든 어쨌든
누군가의 앞에서 시연해봐야 한다. 자신이 심취해서 만든 프레젠테이션 슬라이드들은
자기가 보기엔 완벽해 보일지 모른다. 하지만 다른 사람이 봐 줘야, 어디가 부족한지
알 수 있다. 리허설 할 때의 청중은 자신의 연구를 ‘잘 아는 사람’과 ‘잘 모르는
사람’이 둘 다 있으면 좋다. 잘 모르는 사람은 어느 부분이 이해 안 가는지를 짚어 줄
수 있을 것이고, 잘 아는 사람은 어떻게 그 연구를 효과적으로 전달할 수 있을지
토의해 줄 수 있기 때문이다. 이러한 조언들까지 다 반영해서 프레젠테이션 파일을
만들면, 구두 발표는 준비 끝!이다. 슬라이드를 작성할 때의 세세한 요령들은
\ref{sec:PPT}에서 더 언급하도록 하겠다.

\subsubsection{포스터 발표}
일단 포스터를 만드는 방법부터 이야기해야겠다. 포스터 역시 프레젠테이션 파일로
만든다. 파워포인트의 디자인 - 페이지 설정을 찾아 들어가서, 슬라이드 크기를 사용자
지정으로 해 놓고, 너비와 높이를 각각 84cm, 120cm로 설정한 후 슬라이드 방향을
세로로 바꾼다! 잘 못 찾겠으면, 선배에게 예전에 만들었던 포스터 파일 하나만 달라고
해라. 그러면 이미 용지 설정은 끝나 있을 테니 거기에 작성을 시작하면 된다.  최근
포스터 사이즈가 다양화되고 있다. 정사각형 모양의 포스터를 받는 경우도
있다. 학회에 따라 용지 크기는 맞춰서 잘 설정하도록 하자.

포스터에 들어갈 내용은 구두 발표와 유사하다. 역시 Introduction, Method, Result,
Discussion, Summary가 들어가는 것이 보통이고 거기에 Abstract까지 있으면
금상첨화. 포스터 역시 논문과 마찬가지로 ‘읽어야’ 하는 것이기 때문에 초록이
포함되어 있으면 사람들이 연구 내용을 파악하기가 더 쉬울 것이다.

포스터의 크기는 크다. 크면 전지, 작다 싶어도 전지 반만 한 크기의 종이다. 거기에
자신의 연구를 깨알 같은 글씨로 적으면 구구 절절히 적을 수도 있을 거다. 하지만,
듣는 것보다 읽는 것이 더 귀찮기 때문에(...) 가독성 있게 작성하는 것이
좋다. 왠지, 말하는 것이 아니라 쓰는 것이기 때문에 논문처럼 써야할 것 같지만(나는
그랬다...), 포스터도 ‘발표’이기 때문에 오히려 구두 발표와 같은 간결함을 구사할
필요가 있다. 글씨보다는 그림을 많이 사용하고, 한 눈에 들어올 수 있게 그림에
설명을 곁들이는 것도 좋다. 역지사지의 입장으로, 본인이라면 글씨가 빽빽한 포스터를
꼼꼼히 읽고 싶을지를 잘 고려해서 작성할 것.
 
\subsection{PPT 작성 요령}\label{sec:PPT}
구두 발표, 포스터 발표에도 모두 사용되는 프레젠테이션 파일. 작성 요령에 대해
간단히 이야기해 보겠다. 기본적인 사항이므로 염두에 두어 작성하면 큰 도움이 될
것이다.
 
\begin{packed_enum}
\item 구두 발표의 경우, PPT는 어디까지나 발표자를 돕는 도구여야 한다. 화면에
  표시되는 그림을 설명할 때가 아니라면, 청중의 관심은 발표자에 가 있어야
  한다. 청중에게 이야기하는 사람을 쳐다보도록 하는 것이 좋다. 누가 발표하는지도
  모르는 건 너무하니까.
\item 위와 관련된 노력 일환으로, 슬라이드에 너무 많은 글을 적지 않도록
  하라. 슬라이드를 읽느라 발표자의 말은 듣지도 않는 경우가 많다. 이해하기 쉽도록
  핵심적인 사항들을 적어 놓는 것은 좋지만, 너무 많은 글을 넣지는 않도록 하는 것이
  좋다. 그런 의미에서 중요한 그림들을 많이 사용하는 것도 요령 중 하나.
\item 슬라이드를 작성할 때 글씨 크기 좀 키워라. 그림의 선도 두껍게, 축 이름도
  크게, 심볼도 크게. 잘 보이지 않는 글씨나 그림은 없느니만 못하다. 다른 논문에서
  따온 그림이라 축 이름이 너무 작다면, 새로 덧붙여 써라. 큼직큼직하게 보여야
  이해하기도 쉽고, 독자와 청중도 눈살을 찌푸리지 않을 것이다. 참고로 구두 발표의
  경우, 슬라이드 길이의 1/20보다 작은 글씨는 쓰지 않도록 하는 것이 좋다.
\item 가독성에 관한 것인데, 글씨체는 사람의 취향이 있겠지만... 일반적으로 글자의
  너비나 선의 두께가 일정한 글씨체가 가독성은 좋다고 한다. 예를 들어, Times new
  roman보다는 Arial 같은 글씨체가 좋다고 할까? 한글 글씨로 따지자면 궁서체보다는
  고딕체가 좋다는 말이다.
\item 과도하게 다양한 색상을 사용하는 것은 좋지 않다. 기본 색상 한 가지, 중요한
  부분을 표시하는 색상 한두 가지 정도면 충분하다. 과도한 색상 또한 보는 사람으로
  하여금 어지럽게 할 수 있으니 주의.
\end{packed_enum}

\subsection{효과적인 구두 발표를 위한 팁}
구두 발표를 할 때는 누구나 떨린다. 왜 떨리는지 한 번 생각해 볼까? 친한 동료에게
자신의 연구에 대해 이야기한다고 하자. 그러면 떨릴까? 보통은 아닐 거다. 그 동료가
질문을 한다고 해도, 심지어 따진다고 해도 떨리진 않을 것이다. 그런데 발표를 할 때
떨리는 이유는, 청중을 ‘평가자’로 생각하기 때문이다. 내 연구를 평가하는 사람, 내
실적을 평가하는 사람으로 생각하기 때문에 발표를 하는 것이 두렵다. 청중들은 그저
내 이야기를 들어주는 사람이다. 내 연구가 좀 더 나아질 수 있도록 조언을 해 줄 수
있는 사람으로 생각하고 발표에 임하는 것이, (실제로 그들이 평가를 한다 할지라도)
마음이나마 편할 것이다.

앞에서도 언급했듯이 시간을 지키는 것이 아주 중요하다. 학회에서는 심지어 시간이
되면 종을 쳐서 발표를 끝내버리는 수도 있기 때문에, 핵심적인 사항만 시간 내에 말할
수 있도록 준비해야 한다. 그렇다고 많은 내용을 랩 하는 것으로 커버하려고 하면 안
된다. 발표를 하는 이유는 발표에 목적이 있는 것이 아니라 청중에게 전달하는
것이라는 사실을 잊지 마라. 적당한 속도로, 중요한 사항만 전달할 수 있도록
연습하라.

뻣뻣하게 서서 발표하는 것보다 적절한 몸짓과 바디랭귀지도 섞어주면 청중들의
이해도가 높아질 수 있다. (물론 그런 것까지 고려할 정신이 있다면...ㅠㅠ) 그건
그렇고, 눈은 어디다 둘까? 허공을 쳐다보면서 발표하는 것만큼 어색한 장면이
없다. 발표를 하다 보면, 청중 중에서 자신을 지켜보고 있는 눈길이 느껴질
것이다. 그러면 그 고마운 ‘님’을 집중적으로 쳐다보면서 발표를 하면 좋다. 청중을
8자로 훑으며 하는 방법도 있다지만 개인적으로는 떨려서 그럴 정신이 없다. 그저
고마운 ‘님’에게 내 발표를 바치는 수밖에.

발표를 하게 되면, 마이크나 레이저 포인터를 쓰는 경우가 많다. 학회장에 미리 가서
이들의 작동 상황을 잘 파악해 두는 게 좋다. 특히 레이저 포인터 같은 경우, 어디에서
빛이 나오는지 정도는 알고 들어가자. 포인터를 쐈는데 본인 가슴에 저격수 마냥 빨간
점 꽂지 말고! 그리고 포인터 사용도 최소화 하는 것이 좋다. 글에 밑줄을 쫙쫙
긋는다거나, 단어에 동그라미를 몇 번씩 친다거나 하는 건, 보는 사람으로 하여금
정신없게 할 뿐이다. 사람들은 글에 밑줄을 친히 그어주지 않아도 잘 읽는다. 또 한 번
딱 짚어줘도 무슨 단어를 말하고 싶어 하는지 다 안다. 현란한 포인터의 움직임으로
사람들의 마음을 현혹시키는 것은 발표에 전혀 도움이 되지 않는다.

마지막으로, 질문을 환영해라. 앞에서도 말했지만, 질문하는 사람들을 연구를
까기(...) 위해서라기보다는 조언을 해준다는 입장으로 받아들이는 것이 좋다. 질문을
‘방어’해야겠다는 생각보다는 ‘열린 마음’으로 받아들이라는 말이다. 15분 동안
발표하는 것이기 때문에 연구의 세부사항들은 빠지게 마련이다. 그렇기 때문에 이해를
잘 한 청중일수록 질문하고 싶어지는 것이 많다. 그러니까, 질문이 많이 들어온다는
것은 좋은 징조이므로 안심하도록.

질문이 들어왔을 때 태도는 어떻게 해야 할까? 일단 끝까지 듣고 대답해라. 중간에 말
잘라 먹는 것은 발표자와 청중이 아니더라도 사람 사이의 예의에서 어긋나니까. 그리고
보통 학회장에서 질문을 하게 되면, 질문 하는 사람은 마이크가 없는 경우가 많아서
다른 청중들은 그 질문을 잘 못 들을 경우가 많다. 그러니 다른 청중들을 배려하는
차원에서, “이러이러한 질문이었는데, 제 생각에는~”과 같이 대답하는 것이 좋다.

자, 답을 아는 질문이라면 괜찮다. 그런데 진짜 정 모르겠는 질문이 들어왔다면? (이런
경우가 더 많을지 모른다.) 당황하겠지만 잠시 생각을 해 본 후, 긍정적으로
답변하도록. 정 모르겠는 질문이라면, 보통 자신이 연구에 심취하면서 간과한 부분일
가능성이 높다. 충분히 고려해 볼만한 사항일 것이므로, 그 조언을 받아들여 연구
진행시에 보탬이 되도록 하는 것이 좋다.
 
\section{관측 제안서 작성}
천문학과 다른 자연과학 분야의 가장 큰 차이점 중 하나는, ‘관측’ (Astronomical
Observation)을 통한 자료 수집 과정이다. 우주라는 유일한 대상에 대해서 ‘관측’
이라는 과정을 통해서만 이론의 검증, 새로운 천체와 천문 현상에 대한 발견과 이해가
이루어진다. 즉, 천문학에서의 경쟁력은 얼마나 좋은, 얼마나 많은 관측 자료를 확보할
수 있는가로 결정된다고까지 말할 수 있다. 질 좋은 관측 자료의 확보를 위해서는
구경이 큰 망원경과 이를 통해 모은 정보를 잘 기록할 수 있는 좋은 관측 기기가
필요하다. 이런 시설들을 확보하는 데는 엄청난 양의 재원을 필요로 하기 때문에,
대부분 연구 기관 혹은 국가 차원에서 관측 시설의 운영되고 있다.

관측 시설을 사용하기 위해서, 개별 연구자 혹은 대학 연구팀 (특히 서울대 대학원의 연구팀)의 경우, 관측 기기 사용을 위해 관측 제안서를 제출하고, 경쟁을 통해 망원경 시간을 확보하는 방법을 택하게 된다. 한정된 관측 시설을 이용하기 위해서, 세계 각지의 연구 팀에서 엄청난 수의 관측 제안서를 제출하기 때문에, 망원경 시간을 확보하기 위한 경쟁은 굉장히 심하다. 따라서 망원경 시간의 확보를 위해서는 번뜩이는 아이디어와, 자신의 연구 목표와 실력을 어필할 수 있는 관측 제안서를 작성하는 것이 중요하다... OTL..\\

관측 제안서는 대부분 지상 망원경의 경우, 1년의 두 번 (3월, 9월)의 관측 제안서
모집 (Call for Proposal)을, 우주 망원경의 경우 1년에 한 번 (2월 말 경) 관측
제안서를 모집한다. 따라서 이 시기에는 자신의 아이디어를 구체화하여 관측 제안서를
작성할 수 있도록 항시 아이디어를 준비하도록 하자.

관측 제안서의 양식은 망원경과 관측 기기에 따라 다르기 때문에, 자세하게 설명하지는 않겠으나, 대체적으로 관측 과제의 ‘과학적 정당성 (Scientific Justification)’, ‘실현 가능성 (Feasibility)’, ‘관측 계획 (Experimental Design)’ 등의 부분으로 이루어진다. 과학적 정당성과 실현 가능성을 면밀히 검토하고, 관측의 당위성이 잘 드러날 수 있도록 관측 제안서를 작성하여 관측 시간을 확보할 수 있도록 하자.\\

전 세계에 걸쳐서 관측 제안서를 모집하는 망원경의 종류는 아주 다양하다. 그 중 몇
가지 예시를 들어보자. 한국 천문학자들이 '비교적 낮은' 경쟁을 통해 사용할 수 있는
보현산, 레몬산, 소백산 천문대에 대해서 관측 제안서를 모집한다. 세계적으로는
미국의 국립 광학 천문대 (National Optical Astronomy Observatory)에서 관리하는
망원경들 (예, KPNO, CTIO 등), 일본의 Subaru 망원경, 하와이와 칠레에 위치하고 있는
Gemini 망원경 등이 관측 제안을 모집하는 대표적인 망원경들이다. 그 외에도 더 많은
국소 망원경들에 대한 관측 제안도 할 수 있다. 우주 망원경 중에는 허블 우주
망원경을 비롯하여, Chandra X-ray 위성 등에 대한 관측 제안서 모집이 비교적
정기적으로 이루어지고, 그 외 망원경들은 망원경 운용 계획에 따라 관측 제안서를
모집하므로, 각종 망원경의 상황에 대해서 꾸준히 알아보는 것도 중요하다. 아래의
사이트는 앞서 언급한 망원경들의 관측 제안 모집에 관한 안내를 모아놓은
것이다. 이를 자주 확인하도록 하자.

\begin{packed_item}
\item 보현산 천문대 : \url{http://boao.kasi.re.kr/usefacility/apply.aspx}
\item NOAO 관측 제안 안내 : \url{http://ast.noao.edu/observing/proposal-info}
\item Gemini 관측 제안 안내 : \url{http://www.gemini.edu/?q=node/10991}
\item Subaru 관측 제안 안내 : \url{http://www.naoj.org/Observing/Proposals/index.html}
\item HST 관측 제안 안내 : \url{http://www.stsci.edu/hst/proposing/apt}
\item Chandra 관측 제안 안내 : \url{http://cxc.harvard.edu/proposer}
\end{packed_item}

\section{천문학 저널의 종류와 특징}
유럽 중세에는 집안에 돈이 너무 많아서, 마땅히 할 일이 없어서 심심풀이삼아
공부하여 과학자가 된 경우도 많았다고 한다 카더라. 하지만 이건 옛날얘기고, 오늘날
과학을 하는 대부분의 사람들은 현실적인 고민도 함께 해야 한다. 안정적인 수입, 바로
먹고사니즘. 안정적인 수입을 얻기 위해서는 좋은 직장에 취직해야 한다. 그럼 좋은
직장에 자리잡기 위해서는 무엇이 필요할까? 답은 당연히 실적. 천문학 연구자들에게는
논문이 가장 큰 비중을 차지한다. 연구비를 제공하는 사람들은 객관적인 자료를
기준으로 판단하기 때문에, 이를 위해서는 증명된 학술지에 실린 논문이 필요하다. 본
장에서는 이러한 논문이 게재되는 학술지의 종류와 특징에 대해서 간단히 소개하도록
한다.

천문학에서는 아래와 같은 4개의 학술지가 가장 유명하다.
\begin{packed_item}
\item ApJ         (The Astrophysical Journal)
\item AJ          (The Astronomical Journal)
\item MNRAS   (Monthly Notices of the Royal Astronomical Society)
\item A\&A       (Astronomy and Astrophysics)
\end{packed_item}
ApJ와 AJ는 미국에서 발간하고, 전세계 천문학자들이 가장 많이 애용하는 학술지
이다. 대략 ApJ는 관측, 이론, 파장, 모델에 관계없이 모든 천문학 분야가 고르게
실리고, AJ는 관측연구가 더 많이 게재되는 경향이 있다. A\&A는 유럽에서 발간하는
학술지로, 실제로 유럽 사람들의 투고율이 높다. MNRAS는 영국에서 발간하는
학술지로, 게재료가 없는 것이 가장 큰 특징이다. ApJ, MNRAS, A\&A는 1주일에 한번,
AJ는 1달에 1번 정도 출판된다. 일반적으로 학술지에 논문을 게재하려면 한 쪽 당
10만원이 조금 넘는 논문 게재료를 지불해야 한다\footnote{논문이 컬러로 인쇄되는
  경우 더 비싸다!}. 10쪽만 실어도 100만원 이상이다. (최소한 대학원에 있는
동안에는 연구비에서 게재료를 지원해 주니 걱정은 하지 말자.) MNRAS가 게재료를 받지
않는다는 사실을 다시 한 번 생각해보자. 그렇다면 우리나라 천문학자들이 가장
선호하는 학술지는 위 4개 중에서 무엇일까? ApJ일 것으로 추정된다.

SCI(Science Citation Index, 과학인용색인) 등재 학술지라는 것이 있다. 이러한
학술지들은 학술지의 중요성을 판단하는데 기초가 되는 Impact Factor (I.F.)가
부여된다. Impact Factor (피인용지수)는 학술지에 게재된 논문들이 2년 동안 인용된
숫자의 평균을 의미한다. 당연히 이 수치가 높을수록 학술지의 가치가 높게
인정된다. 위에서 언급한 4개의 학술지 중 I.F가 가장 높은 학술지는 ApJ이다. 해마다
다르지만 일반적으로 5~6점대를 형성한다.

이외에도 천문학과 관련된 어느 정도 특별한 목적의 많은 학술지들이 있다. ApJS (The
Astrophysical Journal Supplement Series)는 그림이나 목록과 같이 많은 양의 자료를
포함하는 논문인 경우 보통 이 학술지에 게재한다. I.F.가 10점대 근처로 높다.
ApJ에는 Letter라는 게 포함되어 있다. 통상적으로 다른 천문학자들에게 긴급하게
알리고픈 중요한 연구가 있을 때 Letter를 이용한다. 일반적인 논문은 제출에서 출판될
때까지 6개월 정도가 걸리는데 반해, Letter는 2개월이면 출판될 수 있다. 참고로
MNRAS에도 Letter가 포함되어 있다.  PASP (Publications of the Astronomical
Society of the Pacific)는 일반 천문학 관련 논문도 싣지만, 주로 천문 기기라든가
프로그램과 같은 논문들을 여기에 게재한다. 이 학술지는 최근 들어 I.F가 감소하는
경향이 있는데, 아마도 순수 천문학 관련 논문의 비중이 적어지고 있어서 그렇지
않을까한다. Solar Physics는 태양을 연구하는 천문학자들이 주로 이용하는
학술지이다. 태양계를 연구하는 사람들이 많이 이용하는 Icarus도 있다. 물론 ApJ도
많이 이용하지만 말이다. 최근에는 일본 학술지 PASJ (Publications of the
Astronomical Society of Japan)도 인기가 꾸준히 증가하고 있다.  그렇다면 I.F가
가장 높은 천문학 관련 학술지는 뭐가 있을까? 여러분도 잘 알고 있으리라 생각되는
Science와 Nature가 있겠다. I.F가 평균 30점대이다. 엄청난 인용 지수다. 여기에
버금가는 천문학만의 학술지는 ARA\&A (Annual Review of Astronomy and
Astrophysics)이다. 이 학술지는 말 그대로 일 년에 한번 나오는 review
논문이다. 천문학 어느 특정분야의 초청받은 대가만이 쓰는 review로 우리가 모두
꿈꾸는 논문이 아닐까?
 
자 그렇다면 이쯤에서 우리나라 천문학계의 학술지는 뭐가 있을까 궁금해진다. 먼저
JKAS (한국천문학회지, Journal of the Korean Astronomical Society)가 있다. 천문학
일반을 다루며, 한국천문학회에서 2개월에 한 번씩 영문으로 발간하고 있으며
한국학술진흥재단의 등재학술지이다. SCIE (Science Citation Index Expanded, 준
SCI)에 등재되어 있으며, I.F는 1점대 미만이다. 제출에서 출판까지 비교적 짧은
기간에 이루어 질 수 있다는 장점이 있다.  JASS (한국우주과학회지, Journal of
Astronomy and Space Sciences)는 한국우주과학회에서 주관해서 발행하며,
등재학술지이다. 일반천문학 부분도 싣지만 우주과학에 관련된 부분이 주를
이룬다. 일년에 4번 영문으로 출판된다. 기타 PKAS(천문학논총, Publications of the
Korean Astronomical Society) 등이 있다. 한국어로 일 년에 2번 이상 출판된다.

\starbreak 이와 같이 국제 학술지와 국내 학술지들을 살펴보았는데, 이런 학술지들은
모두 연구자의 제출된 논문이 심사위원(referee, 동료 심사자)의 조언(comment)에 따라
수정되어야만 논문이 게재된다. 그러나 이와는 달리 referee 코멘트가 없는 경우도
비정기적으로 출판되기도 한다. 이에 해당되는 게 proceeding이다.  Proceeding은 각종
학술대회(workshop, conference, symposium 등)에 참가한 전후 출판되는 것으로
대표적으로 ASP Conference Series (APS=Astronomical Society of the Pacific)와 IAU
Symposia Series (IAU=International Astronomical Union) 등이 있다. 우리는 연구를
할 때 위와 같은 학술지들을 참고문헌으로 사용한다. 그런데 문제는 돈을 줘야
인터넷으로 이들을 받아 볼 수 있다. 심지어 본인이 돈을 주고 게재한 학술지에서도
비용을 지불해야 한다. 그러나 다행히 대학이나 연구기관에서는 미리 비용을 지불하고
학생들이 자유롭게 볼 수 있게 하고 있다. 물론 주 학술지가 아니거나 작은 기관이면
개인적으로 사봐야 하는데, 큰 기관에 소속된 지인에게 부탁할 수 있고,
한국천문연구원에 부탁하면 논문을 복사해주는 서비스를 받을 수 있다.  다른 한편으론
astroph라는 천문학 관련 논문 사이트를 참고할 수 있다. 보통 학술지에 게재
승인되더라도 오랜 시간 후(2개월 이상)에 출판되게 됨으로 다른 연구자들이 내 논문을
볼 수 있게 astroph에 올려 홍보도 겸한다. 학술지 승인 후 여기에 게재하기도 하지만
성질 급한(?) 사람은 승인도 되기 전에 여기에 올리기도 한다. 여하튼 여기에 올리거나
보는 것은 공짜이고 매일매일 최근 연구 논문이 올라오니 잘 활용하자.

초입에 말했듯이 학술지에 논문 쓰는 것은 먹고사니즘이다. 잘 먹고 잘 살려면 업적이
중요하다. 업적을 평가하는 기관은 여러분이 공저자로 참여하는 것도 고려하지만,
제1저자(first author)나 교신저자(corresponding author, 학술지와 메일주고 받는
저자)를 훨씬 우선한다. 학생인 신분에선 교신저자는 힘들 수 있으므로 제1저자가 될
수 있도록 적극적으로 행동하자.  마지막으로 논문이 출판되기까지의 과정을 순서대로
늘어본다.

예시는 관측 분야에서의 연구, 논문 출판의 전형적인 순서이다.
\begin{description}
\item \textsf{관측 신청(Proposal) - 관측 - 자료 처리 및 분석 - 논문 초안
    작성(Drafting) - 공저자 회람(Circulation) - 제출(Submission) - 심사(Referee
    comment + revision) - 게재여부 판정(Accept) 및 astro-ph에 올리기 -
    출판(Publish)}
\end{description}

\section{마무리 멘트}
지금으로부터 10년 전, 고등학교 1학년때 ‘적분‘이라는 걸 처음 배웠다. 아직도
기억난다. 그 길다랗게 생긴 기호를 처음 봤을때의 느낌. 이름도 무시무시하게
’인테그레이터’라고 부른단다. 수학에 열등감을 가지고 있던 필자는 $\int$를 보는
순간 두려웠다. 얘는 또 얼마나 어려운 아이 일까... $\log$, $\Sigma$, $\lim$,
$\mathrm{d}/\mathrm{d}x$를 볼 때도 언제나 마찬가지였다. 하지만 지금 보면 별거
아니다. 쉬운 건 아니지만 그렇다고 두려워 할 건 뭔가? 걔는 그냥 그런 애다. 그게
다다. 거기에 부차적인 의미를 부여해서 이렇게 저렇게 생각하는 건 본인의
판단이다. 그렇게 어렵게 배워놓은 수학규칙들은 후에 공부를 하는데 정말 큰 도움이
되었다. 자연현상을 이해하는데, 내 머리로 이해할 수 있도록 이렇게, 저렇게
요리하는데 아주 강력한 도구가 되니까. 이에 대해서는 다른 부차적인 설명이 필요
없으리라 생각한다. 다들 나보단 수학을 잘 할테니까.

연구방법도 마찬가지다. 딱 한번만 이해하면 된다. 처음 대학원에 입학하면 충만한
의욕으로 이것저것 해보지만 제대로 되는건 정말 하나도 없다. IDL, IRAF, LINUX 같은
연구에 필요한 도구들, 마치 수학에서 적분과 같은 것들, 메뉴얼대로 따라해도 내껀 안
된다. 교재들.. 아무리 봐도 모른다. 논문?  10번 읽어도 이해 안된다. 학회발표는
말할 것도 없다. 그렇다고 절대 좌절할 필요는 없다. 그건 그냥 그런거다. 필자도
그랬고, 하늘처럼 보이는 대학원 선배들도 그랬으니까\footnote{교수님들은
  안그랬을지도...}. 의욕적으로 도전했던 일들이 연거푸 실패한다면 누구나 주눅들게
된다. 하지만 중요한건, 그 초반에 가지고 있던 의욕의 불씨를 1년, 2년, 오랫동안
꺼뜨리지 않고 간직하는 것이다. 장기전 이니까. 천문학 한 10년 연구하고 그만둘꺼
아니잖아?

이 메뉴얼은 위에서 언급한 질문에 자연스레 고개를 끄덕인 신입생들에게 읽혀졌으면
좋겠다. 그리고 그 신입생들에게 8개의 소단원에 걸쳐서 언급한 연구방법이 작게나마
도움이 된다면 그것보다 큰 보람은 없을 것이다.

자 이제 다시 시작해보자. 죽기 전에 우주가 어떻게 생겨났고, 또 어떻게 생겨먹었는지
알아내야지~!


