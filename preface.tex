\chapter{추천사 - 아프니까 대학원생이다.}

나는 대학원생이다. 
이 얼마나 설레는 말인가. 
꿈과 열정과 이성으로 넘치는 젊음으로 대학원 생활을 누리는 사람들! \\

그러나 하루하루 밤낮을 가리지 않고 좌충우돌하면서 맨 땅에 헤딩을 하다보면 몸과 마음이 아파진다. 안개 속에서 나아갈 길이 보이지 않는 때도 가끔씩 있다. 아프니까 대학원생이다. 그래도 신난다.\\

패러다임이 변하고 있다. 맨 땅에 헤딩하다보니 개천에 용들이 사라졌다. 이제는 맨 땅에 헤딩하기 전에 헬멧을 써야 한다. 여기 손 주비 군을 비롯한 헌신적인 대학원생들이 모여 대학원생용 헬멧을 준비했다. 이 헬멧을 쓰면 마음껏 좌충우돌하며 어려움을 헤쳐 나갈 수 있을 것이다. 이 헬멧은 완전하지는 않지만 헬멧으로서의 역할을 충분히 할 것이다. 연구를 통해 헬멧은 계속 개선될 것이다.\\

날아라, 대학원생들이여.\\
전 세계에서 자랑스럽게 높이 나는 흑룡이 되기 바란다.   

{\raggedleft{\scshape 이명균} \\ 2012년 3월 13일\par}

\chapter{여는말}
천문학과 대학원 생활 안내서를 펼친 여러분을 환영한다.

 이 길고 긴 안내서는 서울대학교 천문학과 대학원 신입생들이 대학원에 입학하여, 지도 교수님을 선정하는 과정부터 시작해서 대학원 생활에 적응하는 법, 수료와 졸업을 위한 여러 가지 절차들, 그 외에 연구에 도움이 되는 다양한 팁들을 총 망라한 문서이다. 많은 대학원 신입생들이 느낄 법한, 대학원에 첫 발을 내디뎠을 때의 막막한 시간을 최소화하고, 빠르게 적응할 수 있도록 도움을 주기 위해 작성되었다. 뿐만 아니라, 기존 대학원생들의 연구와 대학원 생활에 관한 다양한 정보를 공유하고, 하나의 문서에 모아 기록에 남기기 위함도 있다. 

 이 안내서에 담긴 내용들은 서울대학교 천문학과 대학원생들이 직접 느끼고, 경험한 것들을 바탕으로 한 다양한 노하우를 담고 있다. 그러나 한편으로는 공식적인 문서는 아니기 때문에, 이를 잘 활용하기 위해서는 각자가 필요한 부분을 차근히 읽어보되, 각종 공문서, 행정실, 대학 본부, 지도 교수님 등을 통해 확인하는 과정을 거칠 것을 추천한다. 

 이 안내서는 크게 네 부분으로 이루어져있다. 첫 번째 부분은 ‘대학원 생활 시작’에 관한 것으로 지도 교수 선정 과정부터 각 연구팀 소개, 컴퓨터 세팅, 강의 소개, 그 외에 기타 생활 적응에 관한 내용을 다루고 있다. 두 번째는 ‘대학원 생활’과 관련한 부분이다. 천문학과 대학원에서 있을 다양한 행사들과, 그 구성원으로서 대학원생들의 역할 등에 대해 소개한다. 세 번째 부분은 실제 연구에 관련한 내용들을 담고 있다. 천문학 관련 논문과 서적에 대한 소개 및 이를 정리 및 활용하는 방법들을 소개한다. 또한 실질적으로 천문학 연구에 활용되는 다양한 프로그램들에 대한 간략한 소개와 유용한 팁 들을 공유하고자 한다. 또 천문학회에 대해서, 자신의 연구를 효과적으로 발표하는 요령, 천문학 (특히 관측 천문학)에서 중요한 관측 제안서 작성 등 천문학 연구 방법에 대한 폭넓지만 얕은 소개도 적어놓았다. 마지막 장에서는 졸업에 관련하여 논문 제출 자격 시험 준비와 졸업 논문 주제 제안, 학위 심사를 준비하는 과정들을 설명하고 있다.

 서울대학교 천문학과는 대학원은 소수의 구성원이 가족과 같은 관계를 구축하고 있으며, 하늘과 우주에 대한 열정을 공유하는 훈훈한 분위기를 자랑한다. 처음에 발을 내딛는 신입생들에게는 조금 낯선 분위기일 수 있으나, 적응하고 나면 그 정을 함빡 느끼면서 연구할 수 있는 낙원(?)과 같은 곳이다. 천문학에 대한 열정과 호기심을 해결하는데, 이만한 곳도 없을 것이다. 그러니! 처음엔 조금 적응이 어려워도, 기존 구성원들과 소통해가며 차근히 적응해보자. 그리고 그 적응과 열정 가득한 연구 생활에 이 안내서가 작은 빛줄기가 되기를 기원한다. 여러분의 연구 여정에 성공만이 가득하길 기원한다. Bon Voyage~!

{\raggedleft{\scshape 저자 일동} \\ 2012년 3월\par}


