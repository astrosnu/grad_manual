\epigraph{지도 교수의 말을 잘 들으면\\
          졸업이 빨라진다.}
 {\textit{2011년 신년교례회에서,\\ \textsc{구본철 교수님}}}

\section{논문 제출 자격 시험}
논문제출 자격시험(논자시, QT, Qualifying test. 이하 논자시)은 어디가서 '천문학
박(석)사' 라고 불리는데 부끄럽지 않을만큼 천문학 제반 지식이 있는지 알아보기 위한
시험으로, 우리 학과에서는 졸업 전에 반드시 이 시험을 통과할 것을 요구한다.
논자시는 한 학기 한 번 실시되며, 보통 1학기는 3월 중후반, 2학기는 9월 중후반에
치뤄진다. 논자시는 보통 4문제 정도 나오는 공통문제와 각 선생님께서 한 문제씩
내시며, 그 중 보통 2~4문제를 선택해서 풀게 되는 선택문제로 구성되어 있다.

논자시를 보기 위해서는 일정 학기 이상 등록하고, 일정 학점을 취득해야 한다. 2014년
현재, 석박통합과정 신입생들의 경우는 4학기 이상 등록하고 통산 36학점 이상을
취득해야 논자시 자격이 주어진다. 입학 후 6학기 이내에 논자시를 통과해야 졸업을 할
수 있으니 주의하자. (이러한 규정은 변동될 수 있으니, 논자시를 볼 때가 다가오면
행정실에 다시 한 번 확인을 해보는 것이 좋다.)  또한, 우리 학과는 졸업 디펜스와
논자시를 한 학기에 볼 수 없으므로, 졸업학기 이전에 논자시를 통과해
두어야한다. 논자시는 가급적 빨리 볼 수 있으면 빨리 보는 것을 추천한다. 첫 번째
이유로는, 시간이 지날수록 전에 들었던 교과 내용들을 더 많이 잊기
때문이다. 수료생이 되서는 더 말할 것도 없고, 코스워크 과정이라 하더라도 논자시를
볼 수 있는 자격을 갖게 될 때 쯤은 이미 (논자시가 주로 출제되는) 교과 과목 위주가
아니라 (논자시에 출제 비중이 미미한) 세미나 과목 위주로 시간표가 짜여진다. 그렇기
때문에 논자시가 늦어질수록 이전 교과 과목에서 배웠던 내용을 다시 기억해내기 더
어려워진다. 두 번째 이유로는, 논자시는 합격률 100\%를 보장해주는 시험이 아니라,
탈락자가 나올 수 있는 시험이기 때문이다. 졸업에 임박해서 부랴부랴 논자시를 쳤다가
떨어질 경우, 자칫 졸업계획에 차질이 생기게 된다. 그렇기 때문에 한두번 떨어진다
하더라도 졸업 일정에 차질이 생기지 않도록 미리미리 논자시를 보자.

논자시를 공부하는데 얼마나 시간을 투자하느냐에 대한 정답은 없다. 나름 열심히
공부했어도 떨어지는 경우도 종종 나오고, 심지어 놀랍게도 단 하루만 공부하고
논자시를 합격한 경우도 있다. 각자 자신이 생각하는 자신의 지능지수와, 그간 자신을
따라다녔던 운과, 자신감과, 놀고 싶은 욕구 등을 고려하여 준비 기간을 설정하자 (이
글 쓴 사람의 경우에는 석박사 둘 다 약 5~7일 정도 공부했더니 꾸역꾸역
통과하더라). 단, 앞서 말한대로 논자시는 3월이나 9월 하순에 많이 보게 되는데, 이
기간은 대략 천문학회 1~3주 전으로, 만약 학회 발표를 하게 된다면 학회 발표 준비로
한창 바쁜 시기이다.
\begin{wrapfigure}{r}{0.6\textwidth}
  \vspace{-20pt}
  \begin{center}
    \includegraphics[width=0.48\textwidth]{./Figures/do_not_try_this_home.png}
  \end{center}
  \vspace{-20pt}
  \legend{\textsf{딱 하루만 공부하고 시험보는거...따라하지 말자.}}
  \vspace{-10pt}
\end{wrapfigure}
그렇기 때문에 방학 중 학회 발표가 결정난다면 방학 때부터 부지런히 학회 및 논자시
공부를 미리미리 해두거나, 아니면 논자시를 한 학기정도 후에 보는 등 현명하게
대처하도록 하자. 학회 발표가 없다면, 논자시를 준비한다는 것을 말씀드리면 논자시
공부 기간 동안에는 연구를 소홀히 해도 양해해주시는 교수님도 계시니, 연구와 공부를
병행하기 어려운 학생은 일주일 정도 급하지 않다면 연구를 잠시 내려놓는 것도 나쁘지
않다. 각자 알아서 그 기간 동안 연구와 공부 간의 균형을 현명하게 조절해라.

논자시를 공부할 수 있는 자료는, 1) 우선 교과 강의 때 필기한 노트나 PPT 및
교과서, 2) 학과에서 제공해주는 최근 수년간의 기출문제, 3) 선배들 사이 잘 찾아보면
나오는 과거 선배님들의 기출풀이 및 정리 요약 노트 등이 있다. 가장 이상적인 것은,
그동안 공부했던 모든 교과 과목을 차근차근 착실히 공부하는 것이겠지만... 시간이
너무 많이 걸린다. 그렇기에 어느 정도 전략을 수립해서 시험에 임하는 것이
필요하다.  논자시 전략 및 공부 방법으로 글 쓴 사람이 썼던 방법을 한 가지 예시를
들면, 우선 자신의 지도 선생님이 내시는 선택 문제는 반드시 맞춘다라고 생각하자
(재미있게도 지도 선생님이 내 주신 문제를 틀린 경우가 적지 않게 발생한다. 심지어
자기 연구 분야에서 문제를 내줬는데 못 푼 학생도 있었으며, 작성자 본인도 지도
선생님께서 이틀 전 수업시간에 다뤘던 내용을 틀려 봤다. 그리고 그 후에 일어난 일에
대해서는... 여백이 부족하여 적지 않겠다.).  그 외에 자신의 연구 분야와의 유사성,
공통문제 과목과의 공부 연계성 등을 고려하여 다른 선생님을 세 분 정도 선정하여, 이
분들이 내시는 선택 문제와 공통문제를 목표로 공부하고, 다른 부분은 과감히
포기한다. 목표로 삼은 분야에 대해 과거 기출문제를 살펴보면, 주로 어느 부분에서
문제가 자주 나오는지를 파악할 수 있다 (꼭 그런 것은 아닌데, 보통 초반부에서 많이
나온다). 그리 자주 출제되는 부분들에 대해 노트와 PPT를 가지고 공부한다. 간혹,
선생님의 최근 관심분야에서 문제가 출제되는 경우도 있으므로, 해당 팀원들에게 최근
선생님의 관심사가 어느 천체에 가 있는지를 파악하는 것도 도움이 될 것이다. 논자시
공부 방법에 정답이라는 것은 없고, 각자 공부 스타일 및 그간 학업 성취도에 따라
자신에게 맞는 최적화된 공부법은 다양하게 있을 수 있다. 그렇기에, 이 논자시 공부
방법은 어디까지나 하나의 예시로 받아들이면 되겠다.

하지만 뭐니뭐니해도 논자시를 가장 쉽게 대처하는 방법은, 교과과목 수업을 들을 때
열심히 공부해두는 것이다. 그러므로 신입생 여러분들은 집중해서 수업에 성실하게
임하여, 이후 논자시에서 전부 좋은 결과가 있기를 바란다.
 

\section{학위 논문 주제 제안}
학위를 취득하려면 반드시 논문심사를 통과하여야 하는데 보통 논문 심사를 받기 이전
(박사과정의 경우 같은 학기에 프로포절과 심사를 할 수 없다.)에 논문 프로포절을
하여야 한다. 프로포절은 자신이 어떤 분야의 연구를 수행할 것이다 라는 것을
발표하는 자리이다. 그리고 논문심사는 자신이 수행한 연구를 발표함으로써 자신이
학위를 받을 만한 자격을 갖추었는지를 검증받는 자리이다. 그러므로 프로포절은
되도록 빨리 (논자시를 합격하지 않은 상태여도 된다. 하지만 반드시 지도교수님과
상의 후 시기를 결정하여야 한다.) 하여 교수님들과 동료들의 Comment를 받기도 하고
자신이 이런 연구를 할 것이라는 것을 홍보하는 것이 좋다.

프로포절 시기는 매 학기 초에 행정실 조교님이 메일로 프로포절 신청을 받는다고
알려주신다. 그러니 본인이 프로포절할 준비가 되었다 싶으면 그때 신청하면
된다. 보통 기한은 개강 후 3주 이내 (3월 혹은 9월 마지막 주나 그 전 주)
이다. 프로포절은 앞서 말했다시피 나는 앞으로 어떤 연구를 수행할 것이다! 라는 것을
발표하는 자리이므로 연구의 결과나 토의는 들어가지 않는다. 대신
\begin{packed_item}
\item 연구를 수행하게 된 배경 (Introduction)
\item 연구 수행 방법 (Data, Analysis, Method 등)
\item 기대되어 지는 결과 (Expected Result) 
\end{packed_item}
등을 발표하게 된다.  발표시간은 석사의 경우 15분 발표/5분 문답, 박사의 경우 20분
발표/10분 문답이다. 논문 심사에서처럼 발표 후 교수님들과의 Closed Door 문답
세션은 없다.

또한 논문심사처럼 당락의 여부가 있는 것도 아니다. 하지만 주제가 바뀌거나 내용을
많이 바뀌게 되면 지도교수님의 판단 하에 다시 프로포절을 하게 되는 경우도 있다.
발표 시 가장 중요한 포인트는 왜 이 연구가 필요한가를 강조하는 것이다. 그러니 학위
논문 연구 주제를 잡을 때에도 이 점을 항상 염두에 두어야 한다.

보통 프로포절을 하기 전에 초록이나, 발표의 요약본 등을 교수님들께 드리면
좋다. 자신이 발표할 내용에 대한 간략한 소개라고 생각하면 된다. 형식, 분량 등이
정해진 것이 아니며 대략 2$\sim$3페이지 정도면 충분하다. 외국인 교수님들이 계시니
이왕이면 영어가 더 좋다. 제출 시기는 최소 이틀 전으로 생각하면 된다. 일단은 모든
교수님들께 드린다고 생각하면 되고 당연히 먼저 지도교수님께 확인을 받도록 하자.

\section{학위 심사}
대학원 입학이 연구자가 되기위한 첫걸음을 의미한다면, 학위심사는 연구자로서
인정받기위해 반드시 넘어야할 큰 산이라 할수 있다. 이 산을 넘으면 그때부턴 더이상
실수를 남발하는 학생이 아닌 진정한 전문가가 되는것이다. 지금까지는 연습이었고,
이제부터는 실전이다.

이 이야기는 필자가 석사학위심사를 마치고 바로 지도교수님을 찾아갔을때, 교수님께서
해주신 말씀이다. 가슴에 잘 새겨놓도록 하자.
 
대학원에서 학위심사는 크게 석사학위심사와 박사학위심사로 나누어 진다. 필자는 이제
대학원 7학기를 시작하는 박사과정 학생으로, 석사학위심사를 마친 경험은 있지만,
박사학위심사 경험은 없다. 따라서 아래의 내용은 주로 석사학위심사를 기준으로
기술됨을 기억해 주길 바란다. 박사학위심사에 대한 내용은 향후 박사학위를 받으면
업데이트 하기로 하겠다.\footnote{대학원생 Tenure 라는것도 있다.}

학위심사는 좁은 의미에서는 학위심사 당일, 발표하는 20분간의 짧은 순간을
의미하지만 \footnote{사실 20분은 수개월보다도 길게 느껴진다..}, 넓게는 학위심사를
지원하는 순간부터 학위를 받기까지 수개월에 걸친 기간을 포함한다고 이야기 할 수
있다. 본 장에서는 시간의 흐름에 맞춰 대학원생이 해야하는 일들과 노하우등을
기술하도록 한다.
 
\subsubsection{심사를 받기 전에}
학위논문 프로포절에서 발표한 연구가 무르익으면 학위심사를 지원할 수
있다. 지원일자는 매년 다르지만, 대략 전반기는 4월 중순, 후반기는 10월중순정도라고
생각하면 된다. 이 시기에 이메일로 공지가 오면, 프린트해서 지도교수님을 찾아가
상담을 하면 된다. 교수님께서 흔쾌히 ok 하시고 사인해주시면 첫번째 고비는
넘어간거다. 만일 "자네는 좀....", "자네 그때까지 마무리 할 수 있겠어?", "자. 일단
오늘나온 이 논문을 좀 보도록 하지." 이런 반응이 나온다면 본인의 평소행실을 곰곰이
되짚어 보도록 하자.

학위심사에 지원하고, 또 지도교수님이 수락할 정도면 연구내용은 어느정도 틀을
갖추었다고 생각된다. 그와 동시에 해야 할 일은 본인의 연구를 논문형식으로 기술하는
것이다. 사실 학위논문을 쓰는, 말그대로 Writing 하는 일은 최소 몇 개월이 걸리기
때문에 학위심사지원 이전부터 시작하는 경우도 많다. 따라서 정확한 시기는
지도교수님, 선배들과의 상담 후 결정하는것이 좋다.

그동안 열심히 연구한 만큼, 학위논문도 열심히 쓸것이다. 하지만 이공계열 학생들이
워낙 글쓰는 일에 서툰지라.. 진도가 잘 안나간다 \footnote{글을 써본적이
  없으니까.. 물론 통계적으로 봤을때 그렇다는 거다.}. 그래도
초록-도입-방법-결과-논의-결론 으로 이루어지는 논문 초안정도는 최대한 빨리
완성하도록 하자. 늦어도 논문심사 발표 5일 전까지는 심사위원들께 논문초안을 드려야
하기 때문이다\footnote{사실 이 시기에는 눈코뜰새없이 바빠 하루가 일년같다. 아무리
  늦어도 3일전에는 드려야 한다.}. 완성된 초안을 스테플러나 스프링노트로 고정한
후, 심사위원 선생님의 연구실을 찾아가 하나하나 드리면 심사를 받기전에 해야할일은
대략 마무리 된다.

\subsubsection{심사 당일}
학위심사는 학기말에 1번씩, 1년에 2차례 진행되며, 석사의 경우 15분발표, 5분 질문,
박사의 경우 45분발표, 15분 + α 질문으로 구성된다. 석사학위심사는 발표가 20분
내외이므로 여러명의 피심사자가 연달아서 발표하지만, 박사학위심사는 발표시간도
길고, 또 더욱 신중하게 판단해야하는 만큼 하루에 한사람씩만 발표한다. 석사학위
심사는 매년 그 날짜가 거의 일정한데, 통계적으로 6, 12월 15 $\pm$ 5일
정도이다. 미리미리 알아두면 좋다.

발표는 정말 잘해야 한다. 시간을 잘 맞추는건 기본이며, 무엇보다 논리적인 허점이
없어야 한다. 미흡한 발표는 청중으로 하여금 피심사자가 지난 2년간 해왔던 연구의
신뢰도에 의문을 가지게 하기 때문이다. 만일 심사에 통과하지 못하면 그 책임은
오롯이 본인의 몫이다. 게다가 본인의 연구를 끝까지 지원해주고, 심사발표까지
허락해주신 지도교수님께는 큰 결례를 범하는 셈이니... 흠...... 잘하자.

어떻게 하면 발표를 잘 할수 있는가? 이건 정답이 없는 문제이다. 기본적인 요령은
3.6장을 참고하자. 그나마 한가지 다행인 점은, 학위심사 발표 자체는 일반적인
학회에서의 발표와 기본적으로 같다는 사실이다. 본인이 연구한 결과는 청중에게
선보이는 일이니 그럴만도 하다. 본인이 석사학위 심사를 받기 까지 적어도 3번의
학위심사가 있으니, 선배들의 발표를 유심히 보면서 준비하도록 하자. 석사학위 심사의
경우 주로 오전 10시쯤 시작해서 12시쯤 끝나는데, 너무 일찍? 시작해서 그런지
빈자리가 많다. 적어도 신입생들에게는 꼭 필요한 경험이 될테니 반드시 참석하도록
하자.
 
\subsubsection{심사를 받은 후에}
석사학위 피심사자들의 발표가 끝나면 대략 점심시간이다. 심사위원 선생님들은
식사하러 가시기 전에 교수회의실에 모여 회의를 하신다. 이 회의가 운명을
결정한다. 다른 대학원 선후배들은 식사하러 가시는데, 발표한 사람 입장에서야.. 밥이
넘어갈까. 그렇게 회의를 마치고, 식사를 마치고 지도교수님께서 돌아오시면 그때
조용히 찾아가자. 합격을 했건, 불합격을 했건, 여러가지 좋은 말씀을 해주실 것이다.

심사에 통과하지 못했다면 6개월 후에 또 기회가 있으니 너무 걱정하지 않아도
된다. 하루정도 푹 쉬고, 다시 대학원에 처음 입학했던 순간처럼 열심히 노력하면
되는거다. 심사에 통과했다면, 합격했다면, 합격의 기쁨을 잠시 누리는것이
좋다. 왜냐하면 다음날부터 또 학위논문을 써야하기 때문이다. 학위심사에
통과했더라도, 학위논문을 완성하지못하면, 또는 완성도가 부족하다면 학위를 받을 수
없다. 학위논문은 대략 1월 말 정도까지 마무리 지어야 하는데, 제본하는데 소요되는
시간 등을 고려하면 1월 20일까지는 완성하는것을 추천한다.

학위논문에는 '인준지'라는것이 있다. 논문의 가장 앞에 위치하는, 심사를 맡은
선생님들의 자필사인이 포함된 페이지이다. 학위심사에 통과하면 미리 공지된 인준지
양식을 다운받아 3장정도 프린트 한 후, 심사위원 선생님들을 찾아가 사인을 받아야
한다. 학위심사가 끝난 후 며칠내로 받을것을 추천한다. 1월 말에 제본할 시기가 다
되어서 선생님 연구실을 찾아가면, 선생님은 미국으로, 유럽으로, 일본으로 출장가
계신 경우가 많기 때문이다. 이렇게 학위논문 제본까지 마치고 나면 2월에 빛나는
졸업장을 받을 수 있다.


\subsubsection{석박사 통합과정 중간평가}
2011학년도 후기까지는 대학원 입학과정이 크게 석사과정과 박사과정으로 나뉘어져
있었다.  박사과정까지 수학을 원하는 학생들 대부분은 석사과정 졸업 뒤 다시
박사과정 신입생으로 지원하는 형태를 따랐기 때문에, 두 번의 논문자격시험, 두 번의
프로포잘(제안 발표), 그리고 두 번의 피 말리는 디펜스(논문발표)라는 무지막지하게
큰 여섯 개의 산을 넘어야만 “박사”라는 타이틀을 받을 수 있었다(박사 선배님들
존경합니다..!).

위에서 언급한 석사과정+박사과정에서의 6번의 반복과정을
3번(논문자격시험1+프로포잘1+디펜스1 = 3!!)으로 줄일 수 있다는 점에서
석박통합과정으로 전환하는 학생들도 있었다.  석박통합과정은 석사학위가 없는 대신,
별도의 박사과정 선발과정을 거치지 않아도 되며 빠르게 코스웍을 끝낼 수 있다는
장점이 있다.  이 경우 석사 2학기 이상 이수하고, 재학 성적 평점 3.3 이상의
지원조건을 만족해야 했으며 지원서 제출 및 교수님과의 면접 과정에서 선발되어야만
했다!  이렇게 석박통합과정으로 진학할 수 있는 학생 수는 제한적이었으므로 대다수의
학생들은 석사학위 취득 후, 다시 박사과정 신입생으로 지원하는 형태로 박사학위를
받을 수 있었다.

하지만 천문학과의 경우, 2012학년도 전기 입학생부터 석사과정 입학 제도를 폐지하고
학사학위를 지원 자격으로 하는 석박통합과정 입학 제도를 도입하면서, 모든 대학원
신입생들이 별도의 석박통합 선발과정 없이 석박통합과정생으로 수학할 수 있게
되었다.  새로 도입된 석박통합과정 제도에서는 3학기 이상 이수하고 24학점 이상 수료
시 박사과정으로 자동 전환되고 박사 논문자격시험을 볼 수 있는 자격이 생긴다.
이러한 석박통합과정에 입학한 대학원생의 입장에서는 별도의 박사진입 과정이 없다는
점에서 ‘아! 난 이제 별 일이 없다면(?) 박사까지 쭈~욱 갈 수 있겠구나!’ 라는 나름의
안심(?)을 할 수 있었다.

그러나 2013년 1월. 당시 학과장이셨던 채종철 교수님께서 2012년 전기 ~ 2013년 전기
석박통합과정 학생들을 소집한 자리에서 “석박사 통합과정 중간 평가”라는 새로운
제도가 도입되었음을 발표하시면서 학생들의 안심은 빛의 속도로 사라지게 된다.  이
제도는 석박통합과정이 기존 석사과정·박사과정에서와 달리 중간 박사진입 과정이나
박사 과정 신입 지원과정이 없기 때문에, 교수님들 입장에서 박사 수학능력이
부족하다고 판단되는 학생들이나 학자로서의 삶이 적성에 맞지 않는 학생들을 가려낼
수 없다는 단점을 메우기 위해 도입되었다.  실제로, 2011년 후기 대학원
입학생들까지는 박사과정 진입 시, 학생이 석사과정동안 보여준 학문적 능력을 통해
지도교수님 혹은 다른 교수님들의 판단 여하에 따라 박사과정으로의 진입의 당락이
결정되었었다.  새로운 석박통합과정 선발제도의 도입으로 이와 같은 박사진입을
원하는 학생의 중간 평가가 시기적으로나 제도적으로 모호해졌기 때문에,
교수님들께서는 2012년 전기 석박통합과정 신입생들부터 “석박사 통합과정 중간
평가”라는 제도를 적용하게 되었다. 자세한 내용은 다음과 같다.

 
1)3인의 평가위원 교수진 구성

석박사 통합과정 중간과정평가 내규에 따르면, 통합과정 학생은 2학기를 이수 한 뒤,
3학기가 시작되기 전까지 지도교수를 포함한 3인 평가위원 지망을 부학부장 교수님
혹은 학과 행정실에 전달한다. 거의 대부분은 자신의 지망대로 3인 교수님이
선정되지만, 학생들의 지망 교수님이 특정 교수님께 겹쳐지는 경우 학과차원에서
조정될 수 있다. 자신의 지도교수님을 제외한 나머지 2분 교수님의 경우, 보통은
자신의 연구 분야와 비슷한 연구를 하고 계시는 교수님을 선정한다.  이렇게 선정된
평가위원회 교수님들은 2년차 평가가 있기 전까지 학생을 예의주시(?) 하시며, 자신의
연구 혹은 학문적 능력을 평가하시는 중요한 분들이니 심사숙고하여 결정하도록 하자!

2) 평가 내용

통합과정 학생이 박사과정으로 진입할 수 있도록 요구되는 평가 내용은 크게 다음 두
가지다.

① 4개 필수과목 이수와 평점 평균 3.3 이상: 통합과정 학생은 4학기까지 1군 필수과목,
천문관측법, 천체물리학, 성간물질, 외부은하와 우주 4개 수업을 필수로 이수해야 하며
필수과목을 포함한 평점 평균이 3.3 이상이 되어야 한다.

② 2년차 연구과제: 석박통합과정 입학 후 2년간 지도교수와 수행한 연구 내용을
평가한다. 학생이 연구자로서의 능력이 있는지, 혹은 잠재된 능력이 있는지를
평가한다.  가장 확실한 방법은 2년 동안 자신이 주저자인 저널 논문을 출판하는
것이며 논문이 없더라도 2년간 수행한 연구 내용을 토대로 학생의 능력을 평가하게
된다.  대학원생에게 학점보다도 더 중요한 것은 바로 연구능력이기 때문에 2년차
연구과제 평가내용은 학생의 평점평균보다도 심사에서 더욱더 비중 있게 다루어진다.

박사과정으로 진입하는 학생의 경우 위의 두 개 내용을 모두 통과해야하며, 모두
만족하지 못하는 경우에는 즉시 중도 탈락된다고 내규에 명시되어 있다.  내용 ①을
만족하는 반면, 내용 ②을 만족하지 못하는 경우 역시 중도탈락 대상이 되며, 그 반대의
경우는 평가위원회의 회의를 거쳐 중도탈락 여부가 결정된다.  학생의 학점 평점보다도
연구능력이 더 중요한 평가대상임을 보여주는 중요한 대목이다.

3) 평가 시기와 형태

① 연구결과물 제출: 통합과정 학생은 4번째 학기가 종료되기 2주 전 까지 (보통 11월
셋 째 주 ~ 넷 째 주) 연구 결과물(학생이 주저자로서 학술지에 제출, 게재승인, 혹은
출판된 논문 원고 또는 연구보고서)을 작성하여 평가위원 교수님들께 제출해야
한다. 앞에서 언급한대로 중간과정 통과의 가장 확실한 방법은 자신의 이름으로 낸
논문을 연구결과물로 제출하는 것이다!  학점 평점의 경우 학과가 그 정보를 조회할 수
있으므로 별도의 서류제출은 필요하지 않다.

② 연구결과 구두발표: 통합과정 학생은 4번째 학기가 종료되기 1주 전까지(보통 12월
첫 째 주) 연구 결과를 평가 위원회에 구두 발표하여 2년차 연구과제에 대한
합격/불합격 여부를 평가 받는다.  정확한 시기는 보통 구두발표 2주 전에 학과
행정실에서 별도로 공지해준다.  시기상 수업의 기말고사와 겹치기 때문에 학생들
차원에서 발표 시기 조정에 대해 건의했지만, 학기말이기 때문에 교수님, 학생들의
출장이 잦고 보다 학과차원의 공식적인 행사인 학위발표가 있기 때문에 현재 내규에서
지정한 일정을 변경하기 힘들다는 답변이 있었다. 모쪼록 중간평가를 치루는 학생들은
연구결과 구두 발표와 기말고사 준비를 적절히(?) 잘 분배하여 진행해야 할 것이다.
내규에 따르면 구두발표는 3인의 평가위원회가 참석한 자리에서 수행한다고 명시되어
있지만, 실제로는 학과 대학원생들과 평가위원회를 제외한 교수님들이 모두 참석한
자리에서 진행된다.  어찌 보면 ‘중간평가 연구결과 구두발표’라는 이름을 하고
있지만, 12분 구두발표 3분 질의응답의 석사학위 디펜스와 같은 형태를 하고 있다.
발표 내용은 본인이 2년간 수행한 연구내용을 충실하게 담고 있어야 하며, 본인의
연구능력을 충분히 어필할 수 있도록 구성되어야 한다.

③ 별도의 질의응답: 구두발표의 경우 공식적으로 개인당 15분으로 시간이 제한되어
있다. 사실상 12분 구두발표 후 3분 동안 교수님들과 학과 대학원생들의 질문에
대답하는 시간을 갖게 되는 것인데, 3분이라는 시간은 평가 위원회 교수님들께서
학생이 자신의 연구에 대한 이해를 잘 하고 있는 것인지 평가하기 어렵다.  따라서
평가위원회 교수님들께서 질문 시간이 부족했다고 느끼시는 경우, 별도의 질의응답
시간을 갖게 된다.

별도의 질의응답은 평가위원회 교수님들 세분과 평가 받는 학생 본인만 참석한
자리에서 이루어지게 된다.  평가위원회 교수님들께서 구두발표 때 질문하지 못하셨던
부분을 질문하기도 하시고, 때로는 학생이 자신의 연구에 대해 잘 이해하고 있는지
확인하기 위해 기초적인 부분들을 질문하시기도 하신다.  이 때 질의응답 시간이
10~15분 정도로 구두발표 시간과 맞먹는 정도이므로 자신의 연구에 대한 이해와 연구
능력을 충분히 어필할 수 있어야한다.  연구 자체도 중요하지만, 특히 자신의 연구에
있어서 기초가 되는 부분들을 확실하게 짚어 놓도록 하자.

4) 평가 결과 발표

평가위원회가 회의를 통해 결정한 내용은 4학기가 끝난 뒤, 지도교수를 통해 학생에게
전달된다.  보통은 심사 다음날 혹은 그 다음 월요일 즈음 알 수 있다. 학과차원에서
별도의 공지를 하지 않으니 결과가 궁금하다면 지도교수님께 넌지시 여쭤볼 수 있다.

